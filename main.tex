%!TEX program = xelatex
\documentclass{article}

\usepackage{geometry}
\geometry{a4paper}
\geometry{includeheadfoot, hmargin=1.5cm, vmargin=1.5cm}

\usepackage{hyperref}

\usepackage{multicol}
\usepackage{changepage}
\usepackage{physics}

\usepackage{amsmath}
\usepackage{amssymb}
\usepackage{amsthm}

% Create nice frames
\usepackage{mdframed}
\mdfsetup{skipabove=\topskip,skipbelow=\topskip}
\mdfdefinestyle{exampledefault}{%
	rightline=true,
	innerleftmargin=10,innerrightmargin=10,
	leftmargin=20,rightmargin=20,
	topline=false,bottomline=false,
	frametitlefont=\bf\large
}
\mdfdefinestyle{exampledefaultcols}{%
	rightline=false,leftline=true,
	innerleftmargin=5,innerrightmargin=5,
	leftmargin=0,rightmargin=0,
	topline=false,bottomline=false,
	frametitlefont=\bf\large
}

% Tables and figures
\usepackage{booktabs}
\usepackage{tabularx}
\usepackage{float}

\usepackage{enumitem}

% Headers and footers
\usepackage{fancyhdr}
\setlength{\headheight}{15.2pt}
\pagestyle{fancy}
\fancyhf{}

\usepackage[warnings-off={mathtools-colon, mathtools-overbracket}]{unicode-math}
\setmathfont{Latin Modern Math}
\setmathfont{TeX Gyre Pagella Math}[range={bb,bbit}, Scale=MatchUppercase]

\usepackage[none]{hyphenat} % Disable hyphenation of long text

%% Headers
\fancyhead[L]{2021 Semester 1}
\fancyhead[C]{Abstract Mathematical Reasoning Exam Notes}
\fancyhead[R]{\thepage}

% Number Sets
\newcommand*{\N}{\mathbb{N}}
\newcommand*{\Z}{\mathbb{Z}}
\newcommand*{\Q}{\mathbb{Q}}
\newcommand*{\I}{\mathbb{I}}
\newcommand*{\R}{\mathbb{R}}
\newcommand*{\C}{\mathbb{C}}

% Table Cell Stretch
\setlength{\tabcolsep}{10pt} % Default value: 6pt
\renewcommand{\arraystretch}{1.5} % Default value: 1

% Set symbols
\let\oldemptyset\emptyset
\let\emptyset\varnothing
\DeclareMathOperator{\setunion}{\cup} % defined for clarity/consistency
\DeclareMathOperator{\setintersection}{\cap}

% Quotation in math mode
\newcommand{\quo}[1]{\text{``}#1\text{"}}

% Exponential e
\newcommand{\e}{e}

\newcommand{\contradiction}{
    \hspace{-1em}
	{\hbox{
	\setbox0=\hbox{$\mkern-3mu{\times}\mkern-3mu$}
	\setbox1=\hbox to0pt{\hss\copy0\hss}
	\copy0\raisebox{0.5\wd0}{\copy1}\raisebox{-0.5\wd0}{\box1}\box0}}
}

%% Theorem environments
\theoremstyle{plain}
\newtheorem{theorem}{Theorem}[section]
\numberwithin{theorem}{subsection}

\theoremstyle{definition}
\newtheorem{definition}{Definition}[section]
\numberwithin{definition}{subsection}

\theoremstyle{remark}
\newtheorem{note}{Note}[section]
\numberwithin{note}{subsection}

\newtheorem*{statement}{Statement}

%% Title page
\title{\textbf{Abstract Mathematical Reasoning}
    \texorpdfstring{\\}{ } {\large Exam Notes}
    \texorpdfstring{\\}{ } {\normalsize 2021, Semester 1}}
\author{
    Rohan Boas \and Tarang Janawalkar \and Oliver Strong
}
\date{}

\usepackage[
    type={CC},
    modifier={by},
    version={4.0},
    imagewidth={5em},
]{doclicense}

\begin{document}

\begin{titlepage}
\maketitle
\thispagestyle{empty}
\vfill
\begin{adjustwidth}{4cm}{4cm}
{\large\bf Preface}

This document was initially created in the week preceeding the exam,
hastily cobbled together by the three authors, Rohan Boas, Tarang Janawalkar and Oliver Strong.
We hope this document might be of some use as
exam notes or as a succinct reference for the subject.
Huge thanks to the lecture notes and unit guide.
This is by no means a replacement for either,
and we greatly appreciate the support and teachings we have recieved.
\end{adjustwidth}
\vfill
\begin{adjustwidth}{5cm}{5cm}
\doclicenseThis
\end{adjustwidth}
\vfill
\end{titlepage}

\newpage

\section{Logic}
\begin{table}[H]
	\centering
	\begin{tabular}{c >{$}c<{$} | c}
			\textbf{Name} & \text{\textbf{Symbol}} & \textbf{Truth Table} \\
			\midrule
		    Negation (NOT) & \neg &
		    \begingroup
		    \renewcommand{\arraystretch}{1}
		    \begin{tabular}{c|c}
		        $P$ & $\neg{P}$ \\
		        \midrule
		        {\sffamily{T}} & {\sffamily{F}} \\
		        {\sffamily{F}} & {\sffamily{T}}
		    \end{tabular}
		    \endgroup
		    \\
		    Conjunction (AND) & \land &
		    \begingroup
		    \renewcommand{\arraystretch}{1}
		    \begin{tabular}{c c|c}
		        $P$ & $Q$ & $P\land Q$ \\
		        \midrule
		        {\sffamily{T}} & {\sffamily{T}} & {\sffamily{T}} \\
		        {\sffamily{T}} & {\sffamily{F}} & {\sffamily{F}} \\
		        {\sffamily{F}} & {\sffamily{T}} & {\sffamily{F}} \\
		        {\sffamily{F}} & {\sffamily{F}} & {\sffamily{F}}
		    \end{tabular}
		    \endgroup
		    \\
		    Disjunction (OR) & \lor &
		    \begingroup
		    \renewcommand{\arraystretch}{1}
		    \begin{tabular}{c c|c}
		        $P$ & $Q$ & $P\lor Q$ \\
		        \midrule
		        {\sffamily{T}} & {\sffamily{T}} & {\sffamily{T}} \\
		        {\sffamily{T}} & {\sffamily{F}} & {\sffamily{T}} \\
		        {\sffamily{F}} & {\sffamily{T}} & {\sffamily{T}} \\
		        {\sffamily{F}} & {\sffamily{F}} & {\sffamily{F}}
		    \end{tabular}
		    \endgroup
		    \\
		    Conditional (IF-THEN) & \implies &
		    \begingroup
		    \renewcommand{\arraystretch}{1}
		    \begin{tabular}{c c|c}
		        $P$ & $Q$ & $P\implies Q$ \\
		        \midrule
		        {\sffamily{T}} & {\sffamily{T}} & {\sffamily{T}} \\
		        {\sffamily{T}} & {\sffamily{F}} & {\sffamily{F}} \\
		        {\sffamily{F}} & {\sffamily{T}} & {\sffamily{T}} \\
		        {\sffamily{F}} & {\sffamily{F}} & {\sffamily{T}}
		    \end{tabular}
		    \endgroup
		    \\
		    Equivalence (IFF) & \iff \text{ or } \equiv &
		    \begingroup
		    \renewcommand{\arraystretch}{1}
		    \begin{tabular}{c c|c}
		        $P$ & $Q$ & $P\iff Q$ \\
		        \midrule
		        {\sffamily{T}} & {\sffamily{T}} & {\sffamily{T}} \\
		        {\sffamily{T}} & {\sffamily{F}} & {\sffamily{F}} \\
		        {\sffamily{F}} & {\sffamily{T}} & {\sffamily{F}} \\
		        {\sffamily{F}} & {\sffamily{F}} & {\sffamily{T}}
		    \end{tabular}
		    \endgroup
	\end{tabular}
	\caption{Boolean operators in order of precedence.}
	\label{tab:Logic Symbols}
\end{table}
\begin{theorem}[de Morgan's Law]
    $\neg{\left(P \land Q\right)} \iff \neg{P} \lor \neg{Q}$.
\end{theorem}
%
\begin{definition}[Tautology]
	A tautology is a statement that is always true.
\end{definition}
%
\begin{definition}[Paradox]
	A paradox is a statement that is always false.
\end{definition}
%
\begin{definition}[Converse]
	The converse of the statement $P\implies Q$	is the statement $Q\implies P$.
\end{definition}
%
\begin{definition}[Contrapositive]
	The contrapositive of $P\implies Q$	is the statement $\neg{Q}\implies \neg{P}$.
\end{definition}
%
\section{Sets}
\begin{table}[H]
	\centering
	\begin{tabular}{c >{$}c<{$} | >{$}c<{$}}
	    \textbf{Name} & \text{\textbf{Symbol}} & \text{\textbf{Definition}} \\
	    \midrule
		Empty set      & \emptyset                & \text{A set with no elements:} \left\{\right\} \\
		Intersection   & \setintersection         & S \setintersection T = \left\{x:x\in S \land x \in T\right\} \\
		Union          & \setunion                & S \setunion T = \left\{x:x\in S \lor x \in T\right\} \\
		Set difference & \backslash \text{ or } - & S - T = \left\{x:x\in S \land x \notin T\right\} \\
		Subset         & \subset                  & \\
		Strict subset  & \subseteq                & \\
		Complement     & \overline{S}  & \text{If } S \subseteq T, \quad \overline{S}=T \backslash S
	\end{tabular}
	% \caption{Set symbols}
	\label{tab:Set Symbols}
\end{table}
%
\begin{definition}[Disjoint]
    If $S \setintersection T = \emptyset$, then $S$ and $T$ are disjoint
    (i.e. two sets are disjoint if their intersection is empty).
\end{definition}
%
\subsection{Construction of the Natural Numbers}
\begin{definition}[$\quo{0}$]
	$\quo{0}$ is defined as the empty set, $\emptyset$ or $\left\{\right\}$, which is the first element of the natural numbers.
\end{definition}
%
\begin{definition}[Successor function]
	The successor function, $S(n)$, is defined as $n \setunion \{n\}$.
\end{definition}
%
\begin{definition}[Addition]
	Addition ($+$) is defined recursively by the following axioms:
	\begin{enumerate}[label={(A\arabic*)}, leftmargin=3.5em, itemsep=0.2em, topsep=0.35em]
		\item $\mathrm{\quo{a} + \quo{0} = \quo{a}}$
		\item $\mathrm{\quo{a} + S(\quo{b}) = S(\quo{a} + \quo{b})}$
	\end{enumerate}
\end{definition}
%
\begin{definition}[Multiplication]
	Multiplication ($\cdot$) is defined recursively by the following axioms:
	\begin{enumerate}[label={(M\arabic*)}, leftmargin=3.5em, itemsep=0.2em, topsep=0.35em]
		\item $\mathrm{\quo{a} \cdot \quo{0} = \quo{0}}$
		\item $\mathrm{\quo{a} \cdot S(\quo{b}) = \quo{a} + (\quo{a} \cdot \quo{b})}$
	\end{enumerate}
\end{definition}
%
\begin{definition}[Exponentiation]
	Exponentiation ($\uparrow$) is defined recursively by the following axioms:
	\begin{enumerate}[label={(E\arabic*)}, leftmargin=3.5em, itemsep=0.2em, topsep=0.35em]
		\item $\mathrm{\quo{a} \uparrow \quo{0} = \quo{1}}$
		\item $\mathrm{\quo{a} \uparrow S(\quo{b}) = \quo{a} \cdot (\quo{a} \uparrow \quo{b})}$
	\end{enumerate}
\end{definition}
%
\subsubsection{Properties of Natural Numbers}
\begin{description}[style=sameline]
	\item[Commutativity of addition]
		$\mathrm{ \quo{a} + \quo{b} = \quo{b} + \quo{a}}$
	\item[Associativity of addition]
		$\mathrm{(\quo{a} + \quo{b}) + \quo{c} = \quo{a} + (\quo{b} + \quo{c})}$
	\item[Commutativity of multiplication]
		$\mathrm{\quo{a} \cdot \quo{b} = \quo{b} \cdot \quo{a}}$
	\item[Associativity of multiplication]
		$\mathrm{(\quo{a} \cdot \quo{b}) \cdot \quo{c} = \quo{a} \cdot (\quo{b} \cdot \quo{c})}$
	\item[Distributivity of multiplication over addition]
		$\mathrm{\quo{a} \cdot  (\quo{b} + \quo{c}) = (\quo{a} \cdot \quo{b}) + (\quo{a} \cdot \quo{c})}$
	\item[Cancellation law]
		If $\mathrm{\quo{a} \cdot \quo{b} = \quo{a} \cdot \quo{c}}$
		and $\mathrm{\quo{a} \ne \quo{0}}$
		then $\mathrm{\quo{b} = \quo{c}}$
\end{description}
%
\subsection{Cartesian Product}
\begin{definition}[Cartesian product]
	The Cartesian product of two sets $S$ and $T$ is the set
	$S\times T$ consisting of all ordered pairs $(x, y)$ where
	$x \in S$ and $y \in T$.
\end{definition}
%
\subsection{Quantifiers}
\begin{table}[H]
    \centering
	\begin{tabular}{c >{$}c<{$} | >{$}c<{$}}
	    \textbf{Name} & \text{\textbf{Symbol}} & \text{\textbf{Usage}} \\
	    \midrule
	     Universal Quantification (For All)        & \forall & \forall x:P(x) \\
	     Existential Quantification (There Exists) & \exists & \exists x:P(x) \\
    \end{tabular}
    % \caption{Quantifiers}
	\label{tab:Quantifiers}
\end{table}
%
\begin{theorem}
$\neg{\left(\forall x:P(x)\right)} \iff \exists x:\neg P(x)$
\end{theorem}
%
\begin{theorem}
$\neg{\left(\exists x:P(x)\right)} \iff \forall x:\neg P(x)$
\end{theorem}
%
\begin{definition}[Greater than ($>$)]
    $x > y \iff \exists a \in \N\backslash\{0\} : x=y+a$
\end{definition}
%
\begin{definition}[Greater than or equal to ($\geqslant$)]
    $x \geqslant y \iff \exists a \in \N : x=y+a$
\end{definition}
%
\section{Relations}
\begin{definition}[Binary relation]
	A binary relation $R$ on $S$ is the subset $S\times S$.
	If $a, \: b \in S$ and  $(a, \: b) \in R$,
	then $aRb$ (``$a$ is related to $b$ under $R$")
\end{definition}
%
\begin{figure}[H]
\begin{mdframed}[style=exampledefault,frametitle={Relation Properties}]
\begin{description}[style=sameline]
	\item[Reflexive]
		$\forall x \in S : xRx$
	\item[Symmetric]
		$\forall x,\: y \in S : xRy \implies yRx$
	\item[Antisymmetric]
		$\forall_\text{distinct} \: x,\: y \in S : xRy \implies \neg (yRx)$
	\item[Transitive]
		$\forall x,\: y, \: z \in S : (xRy \land yRz) \implies xRz$
\end{description}
\begin{definition}[Equivalence relation]
    A relation is an equivalence relation if it is
    reflexive, symmetric, and transitive.
\end{definition}
\end{mdframed}
\end{figure}
%
\begin{definition}[Functions as rules]
    A function $f$ is a rule which maps
    from some set $S$ (the domain)
    to some set $T$ (the codomain)
    (i.e. $f: S \to T$) where each $x \in S$ maps to a unique $f(x) \in T$.
\end{definition}
\begin{definition}[Image]
    The image of a function $f: S \to T$ is the set $f(S)$,
    which is equal to $\{f(x):x \in S\}$.
\end{definition}
\begin{definition}[Functional]
    A relation is functional if
    $\forall x \in S \land \forall y,\: z \in T
        : (xRy \, \land \, yRz) \implies y=z$.
\end{definition}
\begin{note}
    The functional property corresponds to the vertical line test.
\end{note}
\begin{definition}[Left-total]
    A relation is left-total if
    $\forall x \in S : \exists y \in T : xRy$.
\end{definition}
\begin{definition}[Functions as relations]
    A function $f$ is some relation that is functional and left-total.
\end{definition}
%
\begin{figure}[H]
	\begin{mdframed}[style=exampledefault,frametitle={Function Properties}]
	\begin{description}[style=sameline]
		\item[Injective (one-to-one)]
			$\forall x,\: z \in S : \forall y \in T
			: (xRy \,\land\, zRy) \implies x = z$

			Put in words, every element in the codomain ($T$)
			is being mapped to by \textit{at most} one element from the domain ($S$).
		\item[Surjective (onto)]
			$\forall y \in T : \exists x \in S : xRy$

			Put in words, every element in the codomain ($T$)
			is being mapped to by \textit{at least} one element from the domain ($S$).
		\item[Bijective]
			$R$ is injective and surjective.
	\end{description}
	\end{mdframed}
\end{figure}
%
\begin{definition}[Equal functions]
    Two functions are equal if they have the same domain and codomain
    for all $x$, so that $f(x) = g(x)$.
\end{definition}
%
\begin{definition}[Function composition]
    The composition of two functions $f:X \to Y$ and $g:Y \to Z$,
    denoted $g \circ f$, maps $X$ to $Z$, and is defined as
    $(g \circ f)(x) = g(f(x))$.
\end{definition}
%
\begin{theorem}
    For $f:X \to Y$ and $g:Y \to Z$ and  $h:Z \to W$,
    $h \circ (g \circ f) = (h \circ g) \circ f$.
\end{theorem}
%
\begin{definition}[Identity function]
    The identity function of $X$ is $1_X:X \to X$ and is defined as
    $1_X (x)=x$ for all $x \in X$.
\end{definition}
%
\begin{definition}[Inverse function]
    If $f:X \to Y$ and $g:Y \to X$ are functions where
    $g \circ f = 1_X$ and $f \circ g = 1_Y$,
    then $g$ is the inverse of $f$, denoted $f^{-1}$.
\end{definition}
%
\begin{theorem}
    A function $f$ has an inverse if and only if $f$ is bijective.
\end{theorem}
%
\begin{definition}[Equivalence class]
    Let $R$ be an equivalence relation on $S$ and $x \in S$.
    The equivalence class of $x$ is $[x] = \{y \in S : xRy\}$,
    i.e. the set of all elements in $S$ that are related to $x$.

    $x$ is a representative of the equivalence class $[x]$.
    There may be different representations of the same equivalence class.
\end{definition}
%
\section{Number Sets}
\subsection{Rational Numbers}
\begin{definition}[Rational numbers]
    Define an equivalence relation $\sim$ on $\Z \times (\N\backslash\{0\}$
    as $(a,\: b) \sim (c,\: d) \iff ad=bc$.
    The equivalence classes of $\sim$ are rational numbers.
    $[(a,\: b)] = \frac{a}{b}$.
    The set of all rational numbers is
    $\Q = \left\{ \frac{a}{b} : a\in \Z, b\in \N_{>0} \right\}$.
\end{definition}
%
\begin{definition}[Addition on rational numbers]
    $(a,\: b) + (c,\: d) = (ad+bc,\: bd)$
\end{definition}
%
\begin{definition}[Multiplication on rational numbers]
    $(a,\: b) \cdot (c,\: d) = (ac,\: bd)$
\end{definition}
%
\section{Algebraic Structures}
\begin{figure}[H]
	\begin{mdframed}[style=exampledefault,frametitle={Ring Axioms}]
		\begin{definition}[Commutativite ring with identity]
			A commutative ring with identity is a set $S$ with two binary operations $+$ and $\cdot$ such that the following ring axioms are satisfied.
		\end{definition}
		\begin{enumerate}[leftmargin=3.5em, itemsep=0.2em, topsep=0.35em]
			\item[(C1)] $+$ is closed: $\forall a,\:b \in S: a+b \in S$.
			\item[(A1)] $+$ is associative: $\forall a,\:b,\:c \in S: (a+b)+c = a+(b+c)$.
			\item[(A2)] $+$ is commutative: $\forall a,\:b \in S: a+b=b+a$.
			\item[(A3)] additive identity: $\forall a\in S:\exists z \in S:a+z=a$.
			\item[(A4)] additive inverse: $\forall a \in S, \exists b\in S: a+b = z$ (denoted ``0").
			\item[(C2)] $\cdot$ is closed: $\forall a,\:b \in S: a \cdot b \in S$.
			\item[(M1)] $\cdot$ is associative: $\forall a,\:b,\:c \in S:a \cdot (b \cdot c) = (a \cdot c) \cdot c$.
			\item[(M2)] $\cdot$ is commutative: $\forall a,\:b \in S:a \cdot b = b \cdot a$.
			\item[(M3)] multiplicative identity: $\forall a \in S:\exists e \in S:a \cdot e = a$ (denoted ``1").
			\item[(D)] distributivity: $\forall a,\:b,\:c \in S: a \cdot (b + c) = (a \cdot b) + (a \cdot c)$.
		\end{enumerate}
	\end{mdframed}
	\begin{note}
		$\Z$, $\Q$, $\R$, and $\C$ are rings.
	\end{note}
\end{figure}
% \marginpar{\raggedright\footnotesize The definition of rings and fields has been excluded from this document for brevity}
%
\begin{definition}[Integers]
    The set of integers ($\Z$) is the smallest ring containing the natural numbers.
\end{definition}
\begin{note}
    The integers are the closure of the natural numbers in regards to the ring axioms.
\end{note}
%
\begin{definition}[Real numbers]
    Let $D = \Z \times \{\text{sequence of digits $0\ldots 9$}\}$.
    Define an equivalence relation on $D$ such that
    \begin{equation*}
		x_0 . x_1 x_2 \dots x_{k-1} x_k \overline{000} = x_0 . x_1 x_2 \dots x_{k-1} \left(x_k - 1\right)\overline{999}
	\end{equation*}
    The real numbers ($\R$) is the set of the equivalence classes of D using the equivalence relation above.
\end{definition}
%
\begin{definition}[Irrational numbers]
    The irrational numbers ($\I$ or $\overline{\Q}$) are defined as $\R\backslash\Q$.
\end{definition}
%
\begin{figure}[H]
\begin{mdframed}[style=exampledefault,frametitle={Field Axioms}]
	\begin{definition}[Field]
	A field is a set $S$ which is a commutative ring with identity so that it satisfies all ring axioms and also the following axiom.
	\end{definition}
	\begin{enumerate}
		\item[(M4)] multiplicative inverse: $\forall a \in S \ \{0\}:\exists b \in S: a\cdot b=e$.
	\end{enumerate}
\end{mdframed}
\begin{note}
	$\Q$, $\R$, and $\C$ are rings.
\end{note}
\end{figure}
%
\section{Cardinality}
\begin{multicols}{2}
\begin{definition}[Cardinality]
    The cardinality of $X$ ($\#X$ or $\abs{X}$), is the number of elements in $X$.
    If there is a bijection between two sets $X$ and $Y$,
    then $\#X=\#Y$ or equivalently, $\abs{X} = \abs{Y}$.
\end{definition}
%
\begin{definition}[$\N_{<n}$]
    The subset of $\N$ containing all naturals less than $n$.
\end{definition}
%
\begin{definition}[Finite and infinite sets]
    If a set $X$ is such that $\#X=\#\N_{<n}$,
    then the number of elements in $X$ is $n$ and $X$ is a finite set.
    If there is no $n$ such that there is a bijection between $X$ and $\N_{<n}$,
    then $X$ is an infinite set.
\end{definition}
%
\begin{definition}[Countable and uncountable infinities]
    If there exists a bijection between an infinite set $X$ and $\N$,
    then $X$ is countably infinite, else uncountably infinite.
\end{definition}
%
\columnbreak
%
\begin{theorem}$\#\N = \#\Z$\end{theorem}
\begin{theorem}$\#(\N\times\N) = \#\N$\end{theorem}
\begin{theorem}$\#\Q = \#\N$\end{theorem}
\begin{theorem}$\#\N \ne \#\R$\end{theorem}
%
\begin{definition}[Power set]
    The power set of $X$, $\mathscr{P}(X)$, is the set of all subsets of $X$,
    including $\emptyset$ and $X$.
\end{definition}
\begin{theorem}
    The cardinality of $\mathscr{P}(X)$ is $2^{\abs{X}}$.
\end{theorem}
\end{multicols}
%
\section{Proofs}
\begin{figure}[H]
	\begin{mdframed}[style=exampledefault,frametitle={Proof Structure}]
		\begin{enumerate}[leftmargin=3.5em, itemsep=0.2em, topsep=0.35em]
			\item State the proof method that will be used.
			\item If the proof method has conditions show that these conditions have been satisfied before using the proof method.
			\item Clearly state any assumptions and definitions as required by the proof.
			\item Apply the proof method.
			\item Write a sentence summarising the proof method.
			\item Conclude the proof with QED, $\square$, or $\blacksquare$.
		\end{enumerate}
	\end{mdframed}
\end{figure}
%
\begin{note}[Disprove]
    To disprove a statement is to prove its negation.
\end{note}
%
\subsection{Direct Proof}
\begin{enumerate}
    \item Use a definition, i.e. even/odd numbers.
    \item Show LHS = RHS.
    \item For an equivalence ($\iff$) statement, show that both forward ($\implies$) and backward ($\impliedby$) directions are true.
\end{enumerate}
%
\subsection{Truth Table, Tautology, and Paradox}
\begin{enumerate}
    \item Draw a truth table for the statement.
    \item For a proof by truth table, show that the truth tables are the same.
    \item For a proof by tautology, show that the truth table is always true.
    \item For a proof by paradox, show that the truth table is always false.
\end{enumerate}
%
\subsection{Contradiction}
\begin{enumerate}
    \item Assume that the statement is false.
    \item Make a sequence of logical statements that follow from the assumption.
    \item Arrive at a contradiction.
    \item Since we arrive at a contradiction, the original assumption must have been wrong.
    \item Hence the original statement must be true.
    \item The $\contradiction$ and $\implies\!\!\!\!\impliedby$ symbols
            are often used to show a contradiction.
\end{enumerate}
%
\subsection{Contrapositive}
\begin{enumerate}
    \item Write the original statement as a conditional statement.
    \item Rewrite the statement in the contrapositive form.
    \item Try to prove the contrapositive statement, usually a direct proof.
    \item By proving the contrapositive, you have also proved the original statement.
\end{enumerate}
%
\subsection{Contradiction vs. Contrapositive}
For the conditional statement: ``If $P$, then $Q$".

\textbf{By contradiction:} Assume $P$ and $\neg{Q}$, and find some contradiction.

\textbf{By contrapositive:} Assume $\neg{Q}$, and show $\neg{P}$.
%
\subsection{Induction}
\begin{enumerate}
    \item Find the logical statement $P(n)$ so that the theorem can be written in the form: ``Show $P(n)$ is true for all $n \in \N_{\geqslant n_0}$".
    \item Prove the base case is true: $P(n)$ is true for $n=n_0$.
    \item Write the inductive hypothesis: ``Assume $P(n)$ is true for $n=k$".
    \item Write the inductive step: ``$P(n)$ is true for $n=k+1$".
    \item Prove the inductive step is true, usually a direct proof. Remember to utilise the inductive hypothesis.
    \item Conclusion: Since $P(n_0)$ is true, and $P(k+1)$ is true whenever $P(k)$ is true, then the principle of mathematical induction implies $P(n)$ is true for all $n\in \N_{\geqslant n_0}$.
\end{enumerate}
%
\subsection{Strong Induction}
\begin{enumerate}
    \item Find the logical statement $P(n)$ so that the theorem can be written in the form: ``Show $P(n)$ is true for all $n \in \N_{\geqslant n_0}$".
    \item Prove the base case is true: $P(n)$ is true for $n=n_0$.
    \item Write the inductive hypothesis: ``Assume $P(n)$ is true for $n\leqslant k$".
    \item Write the inductive step: ``$P(n)$ is true for $n=k+1$".
    \item Prove the inductive step is true, usually a direct proof. Remember to utilise the inductive hypothesis.
    \item Conclusion: Since $P(n_0)$ is true, and $P(k+1)$ is true whenever $P(n_0),\: \ldots,\: P(k)$ is true, then the principle of strong mathematical induction implies $P(n)$ is true for all $n\in \N_{\geqslant n_0}$.
\end{enumerate}
%
\subsection{Useful Techniques}
\begin{enumerate}
    \item Use the definition of the numbers, i.e. even and odd numbers.
    \item For rational numbers assume that $(a,b)$ is fully simplified, so that $a$ and $b$ are co-prime.
    \item When using strong induction, remember that $P(k), \ldots, P(n_0)$ are assumed to be true.
    \item $a^b=e^{\ln{a^b}}=e^{b\ln{a}}$
    \item $a^k = a^k + b - b$
    \item $a^{k+1} = a \cdot a^k$
    \item $\sum_{i}^{k+1} a_i = a_{k+1} + \sum_{i}^{k} a_i$
\end{enumerate}
%
\section{Sequences}
An infinite sequence on $X$ is a function $a$ with domain $\N_{> 0}$ and codomain $X$. That is, $a:\N_{> 0}\to X$.

For infinite sequences we write $a_n$ and the sequence can be represented as $\left\{a_n\right\}_{n=1}^\infty$, where $n$ is the index of the sequence.
%
\subsection{Convergence}
\begin{theorem}
	A sequence $\left\{a_n\right\}_{n=1}^\infty$ converges to $a\in \R$ if
	\begin{equation*}
		\lim_{n \to \infty}a_n = a \iff \forall \varepsilon > 0 : \exists n_0 \in \N_{> 0}:\forall n \geqslant n_0 : \abs{a_n - a} < \varepsilon
	\end{equation*}
\end{theorem}
%
\begin{theorem}
	A sequence $\left\{a_n\right\}_{n=1}^\infty$ diverges to $+\infty$ if
	\begin{equation*}
		\lim_{n \to \infty}a_n = +\infty \iff \forall N \in \R : \exists n_0 \in \N_{> 0}:\forall n \geqslant n_0 : a_n > N
	\end{equation*}
\end{theorem}
%
\begin{theorem}
	A sequence $\left\{a_n\right\}_{n=1}^\infty$ diverges to $-\infty$ if
	\begin{equation*}
		\lim_{n \to \infty}a_n = -\infty \iff \forall M \in \R : \exists n_0 \in \N_{> 0}:\forall n \geqslant n_0 : a_n < M
	\end{equation*}
\end{theorem}
%
\begin{theorem}[Triangle inequality]
	$\abs{a+b}\leqslant\abs{a}+\abs{b}$.
\end{theorem}
%
\begin{theorem}
	The sequence $\left\{ a_n \right\}_{n=1}^\infty$ converges to $a$ if and only if the sequences $\left\{ a_{2n} \right\}_{n=1}^\infty$ and $\left\{ a_{2n-1} \right\}_{n=1}^\infty$ both converge to $a$.
\end{theorem}
%
\begin{theorem}[Squeeze theorem for sequences]
	Let $\left\{ a_n \right\}_{n=1}^\infty$, $\left\{ b_n \right\}_{n=1}^\infty$, and $\left\{ c_n \right\}_{n=1}^\infty$ be infinite sequences such that $a_n \leqslant b_n \leqslant c_n$ for all $n>n_0$. If $\lim_{n\to\infty}a_n=\lim_{n\to\infty}c_n=d$, then .
\end{theorem}
%
\begin{figure}[H]
	\begin{mdframed}[style=exampledefault,frametitle={Limit Laws for Sequences}]
		\begin{theorem}
			Suppose that $\left\{a_n\right\}$ and $\left\{b_n\right\}$ are convergent infinite sequences on $\R$ that converge to $a$ and $b$ respectively, and let $c\in\R$ be a constant, then
			\begin{enumerate}[label=\normalfont\alph*)]
				\item $\lim_{n\to\infty}c=c$.
				\item $\lim_{n\to\infty}ca_n=ca$.
				\item $\lim_{n\to\infty}\left( a_n + b_n \right)=a+b$.
				\item $\lim_{n\to\infty}\left( a_n - b_n \right)=a-b$.
				\item $\lim_{n\to\infty}\left( a_nb_n \right)=ab$.
				\item $\lim_{n\to\infty}\left( \frac{a_n}{b_n} \right)=\frac{a}{b}$, ($b\neq 0$).
			\end{enumerate}
		\end{theorem}
	\end{mdframed}
\end{figure}
%
\subsection{Cauchy Sequences}
\begin{definition}[Cauchy sequence]
	A sequence $\{a_n\}_{n=1}^\infty$ on $\R$ is a Cauchy sequence if
	\begin{equation*}
		\forall \varepsilon > 0:\exists n_0\in\N_{>0}:\forall n,\: m \geqslant n_0:\abs{a_n-a_m}<\varepsilon
	\end{equation*}
\end{definition}
\begin{theorem}
	A sequence on $\R$ is convergent if and only if is a Cauchy sequence.
\end{theorem}
\subsection{Monotonicity}
%
\begin{definition}[Monotone]
    A sequence is (strictly) monotone if it is
    (strictly) increasing or (strictly) decreasing.
\end{definition}
%
\begin{definition}[Eventually]
    If after removing a finite number of terms from the beginning from a sequence
    it has a certain property, the original sequence has that property eventually.
\end{definition}
%
\begin{definition}[Bounded]
    An infinite sequence is bounded if $L \leqslant a_n \leqslant K$ for all $n$.
\end{definition}
%
\begin{theorem}
    Every bounded eventually monotone sequence converges.
\end{theorem}
%
\subsection{Monotonicty Tests}
\nopagebreak
\begin{multicols}{2}
% \begin{figure}[H]
	\begin{mdframed}[style=exampledefaultcols,frametitle={Difference Test}]
		\begin{theorem}[Difference test]
			A sequence $\{a_n\}^\infty_{n=1}$ is
		\end{theorem}
		\begin{description}[style=sameline]
			\item[Strictly increasing] if $a_{n+1} - a_n > 0$ for all $n$
			\item[Increasing] if $a_{n+1} - a_n \geqslant 0$ for all $n$
			\item[Strictly decreasing] if $a_{n+1} - a_n < 0$ for all $n$
			\item[Decreasing] if $a_{n+1} - a_n \leqslant 0$ for all $n$
		\end{description}
	\end{mdframed}
% \end{figure}
% \begin{figure}[H]
	\begin{mdframed}[style=exampledefaultcols,frametitle={Ratio Test}]
		\begin{theorem}[Ratio test]
			A sequence $\{a_n\}^\infty_{n=1}$ where all $a_n$ are positive is
		\end{theorem}
		\begin{description}[style=sameline]
			\item[Strictly increasing] if $a_{n+1} \div a_n > 1$ for all $n$
			\item[Increasing] if $a_{n+1} \div a_n \geqslant 1$ for all $n$
			\item[Strictly decreasing] if $a_{n+1} \div a_n < 1$ for all $n$
			\item[Decreasing] if $a_{n+1} \div a_n \leqslant 1$ for all $n$
		\end{description}
	\end{mdframed}
	\begin{note}
		If $\lim_{n\to\infty}a_{n+1}\div a_n < 1$,
		then $a_n\to0$ as $n\to\infty$.
	\end{note}
% \end{figure}
% \begin{figure}[H]
	\begin{mdframed}[style=exampledefaultcols,frametitle={Derivative Test}]
		\begin{theorem}[Derivative test]
			If $\{a_n\}^\infty_{n=1}$ is an infinite sequence on $\R$ and
			$a(x)$ is differentiable for $x\in\R_{\geqslant1}$,
			then $\{a_n\}^\infty_{n=1}$ is
		\end{theorem}
		\begin{description}[style=sameline]
			\item[Strictly increasing] if $a'(x) > 0$ for all $x\in\R_{\geqslant 1}$
			\item[Increasing] if $a'(x) \geqslant 0$ for all $x\in\R_{\geqslant 1}$
			\item[Strictly decreasing] if $a'(x) < 0$ for all $x\in\R_{\geqslant 1}$
			\item[Decreasing] if $a'(x) \leqslant 0$ for all $x\in\R_{\geqslant 1}$
		\end{description}
	\end{mdframed}
% \end{figure}
\end{multicols}
%
\section{Series}
\subsection{Infinite Series}
\begin{multicols}{2}
\begin{definition}[Infinite Series]
	An infinite series is an expansion that can be written as
	\begin{equation*}
		\sum_{i=1}^\infty a_i = a_1 + a_2 + a_3 + \cdots
	\end{equation*}
	The numbers $a_i$ are the terms of the series.
\end{definition}
\begin{definition}[Partial Sums]
	The sequence of partial sums $\{s_n\}_{n=1}^\infty$ is defined as
	\begin{equation*}
		s_n = \sum_{i=1}^n a_i
	\end{equation*}
	where $s_n$ is the $n$th partial sum of the infinite series.
\end{definition}
\columnbreak
\begin{definition}[Convergence of series]
	The infinite series $\sum_{i=1}^\infty a_i$ converges to $A$ if the associated sequence of partial sums $\{s_n\}_{n=1}^\infty$ converges to $A$.
\end{definition}
\begin{definition}[Divergence of series]
	The infinite series $\sum_{i=1}^\infty a_i$ diverges to $\pm\infty$, if the associated sequence of partial sums $\{s_n\}_{n=1}^\infty$ diverges to $\pm\infty$.
\end{definition}
\end{multicols}
%
\pagebreak
%
\subsection{Convergence Tests}
For any infinite series of the form $\displaystyle\sum_{i=i_0}^\infty a_i$.
\begin{multicols}{2}
% \begin{figure}[H]
	\begin{mdframed}[style=exampledefaultcols,frametitle={Divergence Test}]
		\begin{equation*}
			\text{Does $\lim_{i\to\infty}a_i\neq0$?}\:
			\begin{cases}
				\text{YES} & \text{$\sum a_i$ Diverges} \\
				\text{NO} & \text{Inconclusive}
			\end{cases}
		\end{equation*}
	\end{mdframed}
% \end{figure}
% \begin{figure}[H]
	\begin{mdframed}[style=exampledefaultcols,frametitle={$p$-Series}]
		\begin{description}[style=sameline]
			\item[Form] $i_0=1$ and $a_i = \frac{1}{i^p}$.
		\end{description}
		\begin{equation*}
			\text{Is $p>1$?}\:
			\begin{cases}
				\text{YES} & \text{$\sum a_i$ Converges} \\
				\text{NO} & \text{$\sum a_i$ Diverges}
			\end{cases}
		\end{equation*}
	\end{mdframed}
% \end{figure}
% \begin{figure}[H]
	\begin{mdframed}[style=exampledefaultcols,frametitle={Geometric Series}]
		\begin{description}[style=sameline]
			\item[Form] $i_0=0$ and $a_i = a r^i$ or $i_0=1$ and $a_i = a r^{i-1}$.
		\end{description}
		\begin{equation*}
			\text{Is $\abs{r}<1$?}\:
			\begin{cases}
				\text{YES} & \text{$\sum a_i$ Converges to $\frac{a}{1-r}$} \\
				\text{NO} & \text{$\sum a_i$ Diverges}
			\end{cases}
		\end{equation*}
	\end{mdframed}
	\begin{note}
		The sequence of partial sums is given by $\displaystyle s_n=a\frac{1-r^{n+1}}{1-r}$.
	\end{note}
% \end{figure}
% \begin{figure}[H]
	\begin{mdframed}[style=exampledefaultcols,frametitle={Alternating Series}]
		\begin{description}[style=sameline]
			\item[Form] $a_i = \left( -1 \right)^i b_i$ or $a_i = \left( -1 \right)^{i+1} b_i$.
			\item[Conditions] $b_i>0$.
		\end{description}
		\begin{equation*}
			\text{Is $b_{i+1}\leqslant b_i$ \& $\lim_{n\to\infty}b_i=0$?}\:
			\begin{cases}
				\text{YES} & \text{$\sum a_i$ Converges} \\
				\text{NO} & \text{Inconclusive}
			\end{cases}
		\end{equation*}
	\end{mdframed}
% \end{figure}
% \begin{figure}[H]
	\begin{mdframed}[style=exampledefaultcols,frametitle={Telescoping Series}]
		\begin{description}[style=sameline]
			\item[Conditions] Subsequent terms cancel out previous terms in the sum.
		\end{description}
		\begin{equation*}
			\text{Does $\lim_{n\to\infty}s_n=s$?}\:
			\begin{cases}
				\text{YES} & \text{$\sum a_i$ Converges to $s$} \\
				\text{NO} & \text{$\sum a_i$ Diverges}
			\end{cases}
		\end{equation*}
	\end{mdframed}
% \end{figure}
% \begin{figure}[H]
	\begin{mdframed}[style=exampledefaultcols,frametitle={Comparison Test}]
		\begin{description}[style=sameline]
			\item[Conditions] Pick $b_i$.
		\end{description}
		\noindent If $\displaystyle \sum_{i=i_0}^\infty b_i$ converges
		\begin{equation*}
			\text{Is $0\leqslant a_i \leqslant b_i$?}\:
			\begin{cases}
				\text{YES} & \text{$\sum a_i$ Converges} \\
				\text{NO} & \text{Inconclusive}
			\end{cases}
		\end{equation*}
		If $\displaystyle \sum_{i=i_0}^\infty b_i$ diverges
		\begin{equation*}
			\text{Is $0\leqslant b_i \leqslant a_i$?}\:
			\begin{cases}
				\text{YES} & \text{$\sum a_i$ Diverges} \\
				\text{NO} & \text{Inconclusive}
			\end{cases}
		\end{equation*}
	\end{mdframed}
% \end{figure}
% \begin{figure}[H]
	\begin{mdframed}[style=exampledefaultcols,frametitle={Limit Comparison Test}]
		\begin{description}[style=sameline]
			\item[Conditions] Pick $b_i$ so that $\displaystyle \lim_{i\to\infty}\frac{a_i}{b_i}=c>0$ \& $a_i,\:b_n>0$.
		\end{description}
		\begin{equation*}
			\text{Does $\sum_{i=i_0}^\infty b_i$ converge?}\:
			\begin{cases}
				\text{YES} & \text{$\sum a_i$ Converges} \\
				\text{NO} & \text{$\sum a_i$ Diverges}
			\end{cases}
		\end{equation*}
	\end{mdframed}
% \end{figure}
% \begin{figure}[H]
	\begin{mdframed}[style=exampledefaultcols,frametitle={Integral Test}]
		\begin{description}[style=sameline]
			\item[Conditions] Let $a_i=f(i)$, so that $f(x)$ is continuous, positive and decreasing on $\left[a,\:\infty\right)$.
		\end{description}
		\begin{equation*}
			\text{Does $\int_a^\infty f(x) \dd{x}$ converge?}\:
			\begin{cases}
				\text{YES} & \text{$\sum_{i=a}^\infty a_i$ Converges} \\
				\text{NO} & \text{$\sum a_i$ Diverges}
			\end{cases}
		\end{equation*}
	\end{mdframed}
% \end{figure}
% \begin{figure}[H]
	\begin{mdframed}[style=exampledefaultcols,frametitle={Ratio Test}]
		\begin{description}[style=sameline]
			\item[Conditions] $\displaystyle \lim_{i\to\infty}\abs{\frac{a_{i+1}}{a_i}}\neq 1$.
		\end{description}
		\begin{equation*}
			\text{Is $\lim_{i\to\infty}\abs{\frac{a_{i+1}}{a_i}} < 1$?}\:
			\begin{cases}
				\text{YES} & \text{$\sum a_i$ Converges Absolutely} \\
				\text{NO} & \text{$\sum a_i$ Diverges}
			\end{cases}
		\end{equation*}
	\end{mdframed}
% \end{figure}
% \begin{figure}[H]
	\begin{mdframed}[style=exampledefaultcols,frametitle={Root Test}]
		\begin{description}[style=sameline]
			\item[Conditions] $\displaystyle \lim_{i\to\infty}\sqrt[n]{\abs{a_i}}\neq 1$.
		\end{description}
		\begin{equation*}
			\text{Is $\lim_{i\to\infty}\sqrt[n]{\abs{a_i}} < 1$?}\:
			\begin{cases}
				\text{YES} & \text{$\sum a_i$ Converges Absolutely} \\
				\text{NO} & \text{$\sum a_i$ Diverges}
			\end{cases}
		\end{equation*}
	\end{mdframed}
% \end{figure}
\end{multicols}
%
\begin{figure}[H]
	\begin{mdframed}[style=exampledefault]
		\begin{theorem}
			Similar to sequences, we can add and subtract convergent infinite series.
		\end{theorem}

		\begin{enumerate}
			\item Suppose $\sum a_i$ and $\sum b_i$ are both convergent infinite series with the same starting index. Then $\sum a_i \pm b_i$ is also convergent.
			\item Suppose $\sum c_i$ is a convergent infinite series and $\sum d_i$ is a divergent infinite series with the same starting index. Then $\sum f_i \pm g_i$ is divergent.
			\item Let $\sum e_i$ be an infinite series and let $d\in\R$ be a constant, and $K\in\N$.
			\begin{enumerate}
				\item $\sum e_i$ and $\sum d e_i$ both converge or both diverge.
				\item $\sum_{i=i_0}^\infty e_i$ and $\sum_{i=K}^\infty e_i$ both converge or both diverge.
			\end{enumerate}
		\end{enumerate}

	\end{mdframed}
\end{figure}
%
\begin{definition}[Absolute Convergence]
	An infinite series $\sum a_i$ converges (or diverges) absolutely if the series of absolute values $\sum a_i$ also converges (or diverges).
\end{definition}
\begin{theorem}
	If an infinite series is absolutely convergent, then it is also convergent.
\end{theorem}
%
\begin{definition}[Conditional Convergence]
	A convergent infinite series that diverges absolutely, converges conditionally.
\end{definition}
\section{Limits of Functions}
\subsection{Limits of Functions on \texorpdfstring{$\mathbb{R}$}{the Reals}}
\begin{definition}[Finite limits using the $\varepsilon$-$\delta$ definition]
    Let a function $f(x)$ be defined for all $x$ in an open interval $I$
    (which contains $x_0 \in \R$), except for $x_0$ which may or may not be defined.
    \begin{equation*}
		\lim_{x\to x_0} f(x) = L \iff \forall\varepsilon>0: \exists\delta>0: \forall x \in I: 0<\abs{x-x_0}<\delta \implies \abs{f(x)-L}<\varepsilon
	\end{equation*}
\end{definition}
%
\begin{figure}[H]
\begin{mdframed}[style=exampledefault,frametitle={Limit Laws for Functions}]
    \begin{theorem} The limit of a sum equals the sum of the limits, i.e.
        \begin{equation*}
              \lim_{x\to x_0}\left(f(x)+g(x)\right)
            = \left(\lim_{x\to x_0}f(x)\right) + \left(\lim_{x\to x_0}g(x)\right)
        \end{equation*}
    \end{theorem}
    \begin{theorem} The limit of a product equals the product of the limits, i.e.
        \begin{equation*}
          \lim_{x\to x_0}\left(f(x)\cdot g(x)\right)
        = \left(\lim_{x\to x_0}f(x)\right) \cdot \left(\lim_{x\to x_0}g(x)\right)
        \end{equation*}
    \end{theorem}
	These are trivially extended to
	division (if the divisor is non-zero),
	subtraction, exponentiation, and moving constants in or outside of a limit, i.e.
	\begin{align*}
    &\lim_{x\to x_0}\left(f(x) - g(x)\right)
        = \left(\lim_{x\to x_0}f(x)\right) - \left(\lim_{x\to x_0}g(x)\right)
    &\qquad
    &\lim_{x\to x_0}\frac{f(x)}{g(x)}
        = \frac{\lim_{x\to x_0}f(x)}{\lim_{x\to x_0}g(x)}
    \\
         &\lim_{x\to x_0}\left(f(x)\right)^n
        = \left(\lim_{x\to x_0}f(x)\right)^n
    &\qquad
    &\lim_{x\to x_0} c \cdot f(x)
        = c \cdot \lim_{x\to x_0}f(x)
	\end{align*}
\end{mdframed}
\end{figure}
%
\begin{definition}[Limits towards $\pm\infty$ using the $\varepsilon$-$\delta$ definition]
    Let a function $f(x)$ be defined for all $x$ in an open interval $I$
    (which contains $x_0 \in \R$), except for $x_0$ which may or may not be defined.
	\begin{equation*}
		\lim_{x\to x_0} f(x) = +\infty \iff \forall M \in \R: \exists\delta>0: \forall x \in I: 0<\abs{x-x_0}<\delta \implies f(x) > M
	\end{equation*}
	\begin{equation*}
    \lim_{x\to x_0} f(x) = -\infty \iff \forall N \in \R: \exists\delta>0: \forall x \in I: 0<\abs{x-x_0}<\delta \implies f(x) < N
	\end{equation*}
\end{definition}
%
\begin{definition}[Limits towards $\pm\infty$]
    Let a function $f(x)$ be defined for all $x$ in an infinite open interval $I$
    extending in the positive $x$ direction.
	\begin{equation*}
		\lim_{x\to\infty} f(x) = L \iff \forall \varepsilon>0: \exists M \in I: x>M \implies \abs{f(x)-L}<\varepsilon
	\end{equation*}
    Let a function $g(x)$ be defined for all $x$ in an infinite open interval $I$
    extending in the negative $x$ direction.
	\begin{equation*}
		\lim_{x\to-\infty} g(x) = L \iff \forall \varepsilon>0: \exists N \in I: x<N \implies \abs{g(x)-L}<\varepsilon
	\end{equation*}
\end{definition}
%
\begin{theorem}
    $\displaystyle\lim_{x\to x_0} f(x)$ exists if and only if
    $\displaystyle\lim_{x\to {x_0}^+} f(x)$ and $\displaystyle\lim_{x\to {x_0}^-} f(x)$
    exist and are equal.
\end{theorem}
%
\begin{theorem}[Squeeze theorem for functions]
    Let an interval $I$ contain $x_0$.
    If $f,\:g,\:h : I \to \R$ are functions such that
    $\forall x \in I\backslash\{x_0\},\; f(x)\leqslant g(x)\leqslant h(x)$
    and
    $\displaystyle\lim_{x\to x_0} f(x) = \lim_{x\to x_0} h(x) = L$,
    then
    $\displaystyle\lim_{x\to x_0} g(x) = L$.
\end{theorem}
%
\begin{note}
   The squeeze theorem holds for left- and right-handed and limits to $\pm\infty$.
\end{note}
%
\begin{definition}[Function continuity using the $\varepsilon$-$\delta$ definition]
    A function $f(x):I\to\R$ is continuous at $c \in I$, if
	\begin{equation*}
		\forall\varepsilon>0: \exists\delta>0: \forall x\in I:0<\abs{x-c}<\delta \implies \abs{f(x)-f(c)}<\varepsilon
	\end{equation*}
    $f(x)$ is continuous on $I$ if $f(x)$ is continuous for all $x \in I$.
\end{definition}
%
\begin{definition}[Function continuity as a sequence]
    A function $f(x):I\to\R$ is continuous at $c \in I$, if
    for every sequence $\{a_n\}^{\infty}_{n=1}$ in $I$ that converges to $c$,
    $\{f(a_n)\}^{\infty}_{n=1}$ converges to $f(c)$.
\end{definition}
%
\begin{note}
    ``$f(x)$ is continuous at $c$'' is equivalent to
    ``$\displaystyle \lim_{x\to c} f(x) = f(c)$''.
\end{note}
%
\begin{note}
    Polynomials are continuous everywhere.
\end{note}
%
\begin{note}
    $\sin$ and $\cos$ are continuous everywhere.
\end{note}
%
\begin{theorem}
    For a function $f$ continuous at $L$
    and another function $g$ where $\displaystyle \lim_{x\to c}g(x) = L$,
    $\displaystyle \lim_{x\to c} (f \circ g)(x) = f(L)$.
\end{theorem}
\begin{note}
    The above theorem still applies when $c$ is $c^\pm$ or $\pm\infty$.
\end{note}
%
\begin{theorem}
    For two functions $f$ and $g$ that are continuous at $c$,
    \begin{enumerate}[label=\normalfont\alph*)]
        \item $f(x) + g(x)$ is continuous at $c$;
        \item $f(x) - g(x)$ is continuous at $c$;
        \item $f(x) \cdot g(x)$ is continuous at $c$; and
        \item $\dfrac{f(x)}{g(x)}$ is continuous at $c$ when $g(c)\ne0$.
    \end{enumerate}
\end{theorem}
%
\begin{theorem}
    For a function $g(x)$ which is continuous at $c$ and
    a function $h(x)$ which is continuous at $g(c)$,
    $h \circ g$ is continuous at $c$.
\end{theorem}
%
\begin{theorem}[L'H\^opital rule]
    For two differentiable (and therefore continuous) functions $f(x)$ and $g(x)$
    except possibly at $x_0$.
    If $\displaystyle \lim_{x\to x_0}f(x)=\lim_{x\to x_0}g(x)=0$,
    or $\displaystyle \lim_{x\to x_0}f(x)=\pm\infty$
        and $\displaystyle \lim_{x\to x_0}g(x)=\pm\infty$,
    then
    $\lim_{x\to x_0}\frac{f(x)}{g(x)} = \lim_{x\to x_0}\frac{f'(x)}{g'(x)}$
    (as long as the limit exists, or diverges to $\pm\infty$).
\end{theorem}
%
\begin{note}
    L'H\^opital's rule also holds for left- and right-handed, and limits that approach $\pm\infty$.
\end{note}
%
\section{Addenda}
\subsection{Assumed Knowledge}
\subsubsection{Factorials}
$\displaystyle n! = \prod_{i=1}^{n} i = 1 \times 2 \times 3 \times \dots \times (n-1) \times n$

\subsubsection{Quadratic Equation}
$\displaystyle x = \frac{-b\pm\sqrt{b^2-4ac}}{2a}$

\subsection{Table of Derivatives}
Let $a\in\R$ be a constant, and $n\in\Z\backslash\{0\}$.
\begin{table}[H]
	\centering
	\renewcommand{\arraystretch}{2.5}
    \begin{tabular}[t]{>{$\displaystyle}c<{$} | >{$\displaystyle}c<{$}}
        f(x) & \dv{f}{x} \\[1em]
		\hline
		a                      & 0                \\
		x                      & 1                \\
		x^n                    & n x^{n-1}        \\
		\e^x                   & \e^x             \\
        \e^{nx}                & n\e^{nx}         \\
        \e^{g(x)}              & g'(x)\e^{g(x)}   \\
		a^x                    & a^x \ln(a)
	\end{tabular}
    \quad
    \begin{tabular}[t]{>{$\displaystyle}c<{$} | >{$\displaystyle}c<{$}}
        f(x) & \dv{f}{x} \\[1em]
		\hline
		\ln{\left(x\right)}     & \frac{1}{x}                            \\
		\log_a{\left(x\right)}  & \frac{1}{x \ln(a)}                     \\
        \sqrt{x}                & \frac{1}{2 \sqrt{x}}                   \\[0.5em]
		\sqrt[n]{x}             & \frac{x^{\frac{1}{n}-1}}{n}            \\
	\end{tabular}
    \quad
    \begin{tabular}[t]{>{$\displaystyle}c<{$} | >{$\displaystyle}c<{$}}
        f(x) & \dv{f}{x} \\[1em]
		\hline
		\sin{\left(x\right)}      & \cos{\left(x\right)}                \\
		\cos{\left(x\right)}      & -\sin{\left(x\right)}               \\
        \tan{\left(x\right)}      & \frac{1}{\cos^2{\left(x\right)}}    \\
	\end{tabular}
\end{table}
\end{document}
