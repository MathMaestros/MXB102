\documentclass{article}
\usepackage[a4paper]{geometry}
% \usepackage{showframe}

\title{MXB102 2021 Semester 1 Exam Notes}
\author{
    Rohan Boas    {\small (n11031743)} \\
    Oliver Strong {\small (n11037580)} \\
    Tarang Janawalkar {\small (n11032201)}
}
\date{\today}

\usepackage{multicol}
\usepackage{changepage}

% \usepackage{mathabx}
\usepackage{amsmath}
\usepackage{amssymb}
\usepackage{amsthm}

% Create nice frames
\usepackage{mdframed}
\mdfsetup{skipabove=\topskip,skipbelow=\topskip}
\mdfdefinestyle{exampledefault}{%
rightline=true,
innerleftmargin=10,innerrightmargin=10,
leftmargin=20,rightmargin=20,
topline=false,bottomline=false,
frametitlefont=\bf\large}

% Tables and figures
\usepackage{booktabs}
\usepackage{tabularx}
\usepackage{float}

\usepackage{physics}

\usepackage{enumitem}

\usepackage[warnings-off={mathtools-colon, mathtools-overbracket}]{unicode-math}
\setmathfont{Latin Modern Math}
\setmathfont{TeX Gyre Pagella Math}[range={bb,bbit}, Scale=MatchUppercase]

% Headers and footers
\usepackage{fancyhdr}
\setlength{\headheight}{15.2pt}
\pagestyle{fancy}
\fancyhf{}

%% Headers
\fancyhead[L]{2021 Semester 1}
\fancyhead[C]{MXB102 Exam Notes}
\fancyhead[R]{\thepage}

\setlength{\tabcolsep}{10pt} % Default value: 6pt
\renewcommand{\arraystretch}{1.5} % Default value: 1

\let\oldemptyset\emptyset
\let\emptyset\varnothing

\DeclareMathOperator{\setunion}{\cup} % defined for clarity/consistency
\DeclareMathOperator{\setintersection}{\cap}

\newcommand{\quo}[1]{\text{``}#1\text{"}}

\theoremstyle{plain}
\newtheorem{theorem}{Theorem}[section]
\numberwithin{theorem}{section}

\theoremstyle{definition}
\newtheorem{definition}{Definition}[section]
\numberwithin{definition}{section}

\newtheorem*{statement}{Statement}
% Usage:
% \begin{statement}
% \end{statement}
% \begin{proof}[Direct Proof]    % (Proof type here)
% \leavevmode
% % proof body
% \end{proof}

\newcommand*{\N}{\mathbb{N}}
\newcommand*{\Z}{\mathbb{Z}}
\newcommand*{\Q}{\mathbb{Q}}
\newcommand*{\I}{\mathbb{I}}
\newcommand*{\R}{\mathbb{R}}
\newcommand*{\C}{\mathbb{C}}

\begin{document}

\maketitle

\section{Logic}
\begin{table}[H]
	\centering
	\begin{tabular}{c >{$}c<{$} | c}
			\textbf{Name} & \text{\textbf{Symbol}} & \textbf{Truth Table} \\
			\midrule
		    Negation (NOT) & \neg & 
		    \begingroup
		    \renewcommand{\arraystretch}{1}
		    \begin{tabular}{c|c}
		        $P$ & $\neg{P}$ \\
		        \midrule 
		        {\sffamily{T}} & {\sffamily{F}} \\
		        {\sffamily{F}} & {\sffamily{T}}
		    \end{tabular}
		    \endgroup
		    \\
		    Conjunction (AND) & \land & 
		    \begingroup
		    \renewcommand{\arraystretch}{1}
		    \begin{tabular}{c c|c}
		        $P$ & $Q$ & $P\land Q$ \\
		        \midrule 
		        {\sffamily{T}} & {\sffamily{T}} & {\sffamily{T}} \\
		        {\sffamily{T}} & {\sffamily{F}} & {\sffamily{F}} \\
		        {\sffamily{F}} & {\sffamily{T}} & {\sffamily{F}} \\
		        {\sffamily{F}} & {\sffamily{F}} & {\sffamily{F}}
		    \end{tabular}
		    \endgroup
		    \\
		    Disjunction (OR) & \lor & 
		    \begingroup
		    \renewcommand{\arraystretch}{1}
		    \begin{tabular}{c c|c}
		        $P$ & $Q$ & $P\lor Q$ \\
		        \midrule 
		        {\sffamily{T}} & {\sffamily{T}} & {\sffamily{T}} \\
		        {\sffamily{T}} & {\sffamily{F}} & {\sffamily{T}} \\
		        {\sffamily{F}} & {\sffamily{T}} & {\sffamily{T}} \\
		        {\sffamily{F}} & {\sffamily{F}} & {\sffamily{F}}
		    \end{tabular}
		    \endgroup
		    \\ 
		    Conditional (IF-THEN) & \implies & 
		    \begingroup
		    \renewcommand{\arraystretch}{1}
		    \begin{tabular}{c c|c}
		        $P$ & $Q$ & $P\implies Q$ \\
		        \midrule 
		        {\sffamily{T}} & {\sffamily{T}} & {\sffamily{T}} \\
		        {\sffamily{T}} & {\sffamily{F}} & {\sffamily{F}} \\
		        {\sffamily{F}} & {\sffamily{T}} & {\sffamily{T}} \\
		        {\sffamily{F}} & {\sffamily{F}} & {\sffamily{T}}
		    \end{tabular}
		    \endgroup
		    \\
		    Equivalence (IFF) & \iff \text{ or } \equiv & 
		    \begingroup
		    \renewcommand{\arraystretch}{1}
		    \begin{tabular}{c c|c}
		        $P$ & $Q$ & $P\iff Q$ \\
		        \midrule 
		        {\sffamily{T}} & {\sffamily{T}} & {\sffamily{T}} \\
		        {\sffamily{T}} & {\sffamily{F}} & {\sffamily{F}} \\
		        {\sffamily{F}} & {\sffamily{T}} & {\sffamily{F}} \\
		        {\sffamily{F}} & {\sffamily{F}} & {\sffamily{T}}
		    \end{tabular}
		    \endgroup
	\end{tabular}
	\caption{Boolean operators in order of precedence.}
	\label{tab:Logic Symbols}
\end{table}
\begin{theorem}[de Morgan's Law]
    $\neg{\left(P \land Q\right)} \iff \neg{P} \lor \neg{Q}$.
\end{theorem}
%
\begin{definition}[Tautology]
	A tautology is a statement that is always true.
\end{definition}
%
\begin{definition}[Paradox]
	A paradox is a statement that is always false.
\end{definition}
%
\begin{definition}[Transitive]
	An operation is transitive if   that is always false.
\end{definition}
%
\begin{definition}[Converse]
	The converse of the statement $P\implies Q$	is the statement $Q\implies P$.
\end{definition}
%
\begin{definition}[Contrapositive]
	The contrapositive of $P\implies Q$	is the statement $\neg{Q}\implies \neg{P}$.
\end{definition}
%
\section{Sets}
\begin{table}[H]
	\centering
	\begin{tabular}{c >{$}c<{$} | >{$}c<{$}}
	    \textbf{Name} & \text{\textbf{Symbol}} & \text{\textbf{Definition}} \\
	    \midrule
		Empty set      & \emptyset                & \text{A set with no elements:} \left\{\right\} \\
		Intersection   & \setintersection         & S \setintersection T = \left\{x:x\in S \land x \in T\right\} \\
		Union          & \setunion                & S \setunion T = \left\{x:x\in S \lor x \in T\right\} \\
		Set difference & \backslash \text{ or } - & S - T = \left\{x:x\in S \land x \notin T\right\} \\
		Subset         & \subset                  & \\
		Strict subset  & \subseteq                & \\
		Complement     & \overline{S}  & \text{If } S \subseteq T, \quad \overline{S}=T \backslash S
	\end{tabular}
	% \caption{Set symbols}
	\label{tab:Set Symbols}
\end{table}
%
\begin{definition}[Disjoint]
    If $S \setintersection T = \emptyset$, then $S$ and $T$ are disjoint
    (i.e. two sets are disjoint if their intersection is empty).
\end{definition}
%
\subsection{Construction of the Natural Numbers}
\begin{definition}[$\quo{0}$]
	$\quo{0}$ is defined as the empty set, $\emptyset$ or $\left\{\right\}$, which is the first element of the natural numbers.
\end{definition}
%
\begin{definition}[Successor function]
	The successor function, $S(n)$, is defined as $n \setunion \{n\}$.
\end{definition}
%
\begin{definition}[Addition]
	Addition ($+$) is defined recursively by the following axioms:
	\begin{enumerate}[label={(A\arabic*)}, leftmargin=3.5em, itemsep=0.2em, topsep=0.35em]
		\item $\mathrm{\quo{a} + \quo{0} = \quo{a}}$
		\item $\mathrm{\quo{a} + S(\quo{b}) = S(\quo{a} + \quo{b})}$
	\end{enumerate}
\end{definition}
%
\begin{definition}[Multiplication]
	Multiplication ($\cdot$) is defined recursively by the following axioms:
	\begin{enumerate}[label={(M\arabic*)}, leftmargin=3.5em, itemsep=0.2em, topsep=0.35em]
		\item $\mathrm{\quo{a} \cdot \quo{0} = \quo{0}}$
		\item $\mathrm{\quo{a} \cdot S(\quo{b}) = \quo{a} + (\quo{a} \cdot \quo{b})}$
	\end{enumerate}
\end{definition}
%
\begin{definition}[Exponentiation]
	Exponentiation ($\uparrow$) is defined recursively by the following axioms:
	\begin{enumerate}[label={(E\arabic*)}, leftmargin=3.5em, itemsep=0.2em, topsep=0.35em]
		\item $\mathrm{\quo{a} \uparrow \quo{0} = \quo{1}}$
		\item $\mathrm{\quo{a} \uparrow S(\quo{b}) = \quo{a} \cdot (\quo{a} \uparrow \quo{b})}$
	\end{enumerate}
\end{definition}
%
\subsubsection{Other properties}
\begin{description}[style=nextline]
	\item[Commutativity of addition]
		$\mathrm{ \quo{a} + \quo{b} = \quo{b} + \quo{a}}$
	\item[Associativity of addition]
		$\mathrm{(\quo{a} + \quo{b}) + ``c`` = \quo{a} + (\quo{b} + ``c``)}$
	\item[Commutativity of multiplication]
		$\mathrm{\quo{a} .  \quo{b} = \quo{b} .  \quo{a}}$
	\item[Associativity of multiplication]
		$\mathrm{(\quo{a} \cdot \quo{b}) \cdot \quo{c} = \quo{a} \cdot (\quo{b} \cdot \quo{c})}$
	\item[Distributivity of multiplication over addition]
		$\mathrm{\quo{a} \cdot  (\quo{b} + \quo{c}) = (\quo{a} \cdot \quo{b}) + (\quo{a} \cdot \quo{c})}$
	\item[Cancellation law]
		If $\mathrm{\quo{a} \cdot \quo{b} = \quo{a} \cdot \quo{c}}$
		and $\mathrm{\quo{a} \ne \quo{0}}$
		then $\mathrm{\quo{b} = \quo{c}}$
\end{description}
%
\subsection{Cartesian Product}
\begin{definition}[Cartesian product]
	The Cartesian product of two sets $S$ and $T$ is the set
	$S\times T$ consisting of all ordered pairs $(x, y)$ where
	$x \in S$ and $y \in T$.
\end{definition}
%
\subsection{Quantifiers}
\begin{table}[H]
    \centering
	\begin{tabular}{c >{$}c<{$} | >{$}c<{$}}
	    \textbf{Name} & \text{\textbf{Symbol}} & \text{\textbf{Usage}} \\
	    \midrule
	     Universal Quantification (For All)        & \forall & \forall x:P(x) \\
	     Existential Quantification (There Exists) & \exists & \exists x:P(x) \\
    \end{tabular}
    % \caption{Quantifiers}
	\label{tab:Quantifiers}
\end{table}
%
\begin{theorem}
$\neg{\left(\forall x:P(x)\right)} \iff \exists x:\neg P(x)$
\end{theorem}
%
\begin{theorem}
$\neg{\left(\exists x:P(x)\right)} \iff \forall x:\neg P(x)$
\end{theorem}
%
\begin{definition}[Greater than ($>$)]
    $x > y \iff \exists a \in \N\backslash\{0\} : x=y+a$
\end{definition}
%
\begin{definition}[Greater than or equal to ($\ge$)]
    $x \ge y \iff \exists a \in \N : x=y+a$
\end{definition}
%
\section{Relations}
\begin{definition}[Binary relation]
	A binary relation $R$ on $S$ is the subset $S\times S$.
	If $a, \: b \in S$ and  $(a, \: b) \in R$,
	then $aRb$ (``$a$ is related to $b$ under $R$")
\end{definition}
%
\begin{figure}[H]
\begin{mdframed}[style=exampledefault,frametitle={Relation properties}]
\begin{description}[style=sameline]
	\item[Reflexive:]
		$\forall x \in S : xRx$
	\item[Symmetric:]
		$\forall x,\: y \in S : xRy \implies yRx$
	\item[Antisymmetric:]
		$\forall_\text{distinct} \: x,\: y \in S : xRy \implies \neg (yRx)$
	\item[Transitive:]
		$\forall x,\: y, \: z \in S : (xRy \land yRz) \implies xRz$
\end{description}
\end{mdframed}
\end{figure}
%
\begin{adjustwidth}{-10pt}{-10pt}
\begin{multicols}{2}
\begin{definition}[Equivalence relation]\ \\
    A relation is an equivalence relation if it is
    reflexive, symmetric, and transitive.
\end{definition}
\begin{definition}[Function (rule ver.)]\ \\
    A function $f$ is a rule which maps
    from some set $S$ (the domain)
    to some set $T$ (the codomain)
    (i.e. $ f: S \to T$) where each $x \in S$ maps to a unique $f(x) \in T$.
\end{definition}
\begin{definition}[Image of a function]\ \\
    The image of a function $f: S \to T$ is the set $f(S)$,
    which is equal to $\{f(x):x \in S\}$.
\end{definition}
\begin{definition}[Functional]\ \\
    A relation is functional if
    $\forall x \in S \land \forall y,\: z \in T
        : (xRy \, \land \, yRz) \implies y=z$.
    
    (N.B. This functional property corresponds to the vertical line test.)
\end{definition}
\begin{definition}[Left-total]\ \\
    A relation is left-total if
    $\forall x \in S : \exists y \in T : xRy$.
\end{definition}
\begin{definition}[Function (relation ver.)]\ \\
    A function $f$ is some relation that is functional and left-total.
\end{definition}
\end{multicols}
\end{adjustwidth}
%
\begin{figure}[H]
\begin{mdframed}[style=exampledefault,frametitle={Function properties}]
\begin{description}[style=sameline]
	\item[Injective (one-to-one):]
		$\forall x,\: z \in S : \forall y \in T
		: (xRy \,\land\, zRy) \implies x = z$
		
		Put in words, every element in the codomain ($T$)
		is being mapped to by \textit{at most} one element from the domain ($S$).
	\item[Surjective (onto):]
		$\forall y \in T : \exists x \in S : xRy$
		
		Put in words, every element in the codomain ($T$)
		is being mapped to by \textit{at least} one element from the domain ($S$).
	\item[Bijective:]
		$R$ is injective and surjective.
\end{description}
\end{mdframed}
\end{figure}
%
\begin{definition}[Equal functions]
    Two functions are equal if they have the same domain and codomain
    for all $x$, so that $f(x) = g(x)$.
\end{definition}
%
\begin{definition}[Function composition]
    The composition of two functions, $f:X \to Y$ and $g:Y \to Z$,
    is denoted $g \circ f$ which maps from $X$ to $Z$ and is defined as
    $(g \circ f)(x) = g(f(x))$.
\end{definition}
%
\begin{theorem}
    For $f:X \to Y$ and $g:Y \to Z$ and  $h:Z \to W$,
    $h \circ (g \circ f) = (h \circ g) \circ f$.
\end{theorem}
%
\begin{definition}[Identity function]
    The identity function of $X$ is $1_X:X \to X$ and is defined as
    $1_X (x)=x$ for all $x \in X$.
\end{definition}
%
\begin{definition}[Inverse function]
    If $f:X \to Y$ and $g:Y \to X$ are functions where
    $g \circ f = 1_X$ and $f \circ g = 1_Y$,
    then $g$ is the inverse of $f$, denoted $f^{-1}$.
\end{definition}
%
\begin{theorem}
    A function $f$ has an inverse if and only if $f$ is bijective.
\end{theorem}
%
\begin{definition}[Equivalence class]
    Let $R$ be an equivalence relation on $S$ and $x \in S$.
    The equivalence class of $x$ is $[x] = \{y \in S : xRy\}$,
    i.e. the set of all elements in $S$ that are related to $x$.
    
    $x$ is a representative of the equivalence class $[x]$.
    There may be different representations of the same equivalence class.
\end{definition}
%
\section{Number Sets}
%
\begin{definition}[Rational numbers]
    Define an equivalence relation $\sim$ on $\Z \times (\N\backslash\{0\}$
    as $(a,\: b) \sim (c,\: d) \iff ad=bc$.
    The equivalence classes of $\sim$ are rational numbers.
    $[(a,\: b)] = \frac{a}{b}$.
    The set of all rational numbers is
    $\Q = \left\{ \frac{a}{b} : a\in \Z, b\in \N_{>0} \right\}$.
\end{definition}
%
\begin{definition}[Addition on rational numbers]
    $(a,\: b) + (c,\: d) = (ad+bc,\: bd)$
\end{definition}
%
\begin{definition}[Multiplication on rational numbers]
    $(a,\: b) \cdot (c,\: d) = (ac,\: bd)$
\end{definition}
%
N.B. The definition of rings and fields has been excluded from this document for brevity (of writing)
%
\begin{definition}[Integers]
    The set of integers ($\Z$) is the smallest ring containing the natural numbers.
    N.B. The integers are the closure of the natural numbers with respect to the ring axioms.
\end{definition}
%
\begin{definition}[Real numbers (infinite decimal ver.)]
    Let $D = \Z \times \{\text{sequence of digits $0\ldots 9$}\}$.
    Define an equivalence relation on $D$ such that
    \begin{equation*}
		x_0 . x_1 x_2 \dots x_{k-1} x_k 000 \ldots = x_0 . x_1 x_2 \dots x_{k-1} \left(x_k - 1\right)999 \ldots
	\end{equation*}.
    The real numbers ($\R$) is the set of the equivalence classes of D using the equivalence relation above.
\end{definition}
%
\begin{definition}[Irrational numbers]
    The irrational numbers ($\I$ or $\overline{\Q}$) is $\R\backslash\Q$.
\end{definition}
%
\section{Cardinality}
\begin{definition}[Cardinality]
    The cardinality of $X$ ($\#X$ or $\abs{X}$), is the number of elements in $X$.
    If there is a bijection between two sets $X$ and $Y$,
    then $\abs{X} = \abs{Y}$ or equivalently, $\#X=\#Y$.
\end{definition}
%
\begin{definition}[$\N_{<n}$]
    The subset of $\N$ containing all naturals less than $n$.
\end{definition}
%
\section{Proofs}
\subsection{Proof Layout}
\begin{enumerate}
    \item State the proof method that will be used.
    \item If the proof method has conditions show that these conditions have been satisfied before using the proof method.
    \item Label all important statements required by the proof, such as assumptions.
    \item Apply the proof method.
    \item Conclude by stating the proof method.
\end{enumerate}
%
\subsection{Direct Proof}
\begin{enumerate}
    \item Use a definition (even/odd numbers).
    \item Show LHS = RHS.
    \item For an equivalence ($\iff$) statement, show that both forward ($\implies$) and backward ($\impliedby$) directions are true.
\end{enumerate}
%
\subsection{Contradiction}
\begin{enumerate}
    \item Assume that the statement is false.
    \item Make a sequence of logical statements that follow from the assumption.
    \item Arrive at a contradiction.
    \item Since we arrive at a contradiction, the original assumption must have been wrong.
    \item Thus the statement is true.
\end{enumerate}
%
\subsection{Contrapositive}
\begin{enumerate}
    \item Write the original statement as a conditional statement.
    \item Rewrite the statement in the contrapositive form.
    \item Try to prove the contrapositive statement, usually a direct proof.
    \item By proving the contrapositive, you have also proved the original statement.
\end{enumerate}
%
\subsection{Contradiction vs. Contrapositive}
\underline{For the conditional statement: ``If $P$, then $Q$".}

\noindent\textbf{By contradiction:} Assume $P$ and $\neg{Q}$, and find some contradiction.

\noindent\textbf{By contrapositive:} Assume $\neg{Q}$, and show $\neg{P}$.
%
\subsection{Induction}
\begin{enumerate}
    \item Find the logical statement $P(n)$ so that the theorem can be written in the form: ``Show $P(n)$ is true for all $n \in \N_{\geq n_0}$".
    \item Prove the base case is true: $P(n)$ is true for $n=n_0$.
    \item Write the inductive hypothesis: ``Assume $P(n)$ is true for $n=k$".
    \item Write the inductive step: ``$P(n)$ is true for $n=k+1$".
    \item Prove the inductive step is true, usually a direct proof. Remember to utilise the inductive hypothesis.
    \item Conclusion: Since $P(n_0)$ is true, and $P(k+1)$ is true whenever $P(k)$ is true, then the principle of mathematical induction implies $P(n)$ is true for all $n\in \N_{\geq n_0}$.
\end{enumerate}
%
\subsection{Strong Induction}
\begin{enumerate}
    \item Find the logical statement $P(n)$ so that the theorem can be written in the form: ``Show $P(n)$ is true for all $n \in \N_{\geq n_0}$".
    \item Prove the base case is true: $P(n)$ is true for $n=n_0$.
    \item Write the inductive hypothesis: ``Assume $P(n)$ is true for $n\leq k$".
    \item Write the inductive step: ``$P(n)$ is true for $n=k+1$".
    \item Prove the inductive step is true, usually a direct proof. Remember to utilise the inductive hypothesis.
    \item Conclusion: Since $P(n_0)$ is true, and $P(k+1)$ is true whenever $P(n_0),\: \ldots,\: P(k)$ is true, then the principle of strong mathematical induction implies $P(n)$ is true for all $n\in \N_{\geq n_0}$.
\end{enumerate}
%
\subsection{Useful Techniques}
\begin{enumerate}
    \item Use the definition of the numbers, i.e., even and odd numbers.
    \item For rational numbers assume that $(a,b)$ is fully simplified, so that $a$ and $b$ are co-prime.
    \item When using strong induction, remember that $P(k-1), \ldots, P(n_0)$ are assumed to be true.
    \item $a^b=e^{\ln{a^b}}=e^{b\ln{a}}$.
\end{enumerate}
%
\section{Sequences}
An infinite sequence on $X$ is a function $a$ with domain $\N_{> 0}$ and codomain $X$. That is, $a:\N_{> 0}\rightarrow X$.

For infinite sequences we write $a_n$ and the sequence can be represented as $\left\{a_n\right\}_{n=1}^\infty$, where $n$ is the index of the sequence.
%
\subsection{Convergence}
\begin{theorem}
	A sequence $\left\{a_n\right\}_{n=1}^\infty$ converges to $a\in \R$ if
	\begin{equation*}
		\forall \epsilon > 0 : \exists n_0 \in \N_{> 0}:\forall n \geq n_0 : \abs{a_n - a} < \epsilon
	\end{equation*}
	where $\lim_{n \rightarrow \infty}a_n = a$.
\end{theorem}
%
\begin{theorem}
	A sequence $\left\{a_n\right\}_{n=1}^\infty$ diverges to $+\infty$ if
	\begin{equation*}
		\forall N \in \R : \exists n_0 \in \N_{> 0}:\forall n \geq n_0 : a_n > N
	\end{equation*}
	where $\lim_{n \rightarrow \infty}a_n = \infty$.
\end{theorem}
%
\begin{theorem}
	A sequence $\left\{a_n\right\}_{n=1}^\infty$ diverges to $-\infty$ if
	\begin{equation*}
		\forall M \in \R : \exists n_0 \in \N_{> 0}:\forall n \geq n_0 : a_n < M
	\end{equation*}
	where $\lim_{n \rightarrow \infty}a_n = -\infty$.
\end{theorem}
%
\begin{theorem}[Triangle Inequality]
	$\abs{a+b}\leq\abs{a}+\abs{b}$.
\end{theorem}
%
\begin{theorem}
	A sequence $\left\{ a_n \right\}_{n=1}^\infty$ converges to $a$ iff the sequences $\left\{ a_{2n} \right\}_{n=1}^\infty$ and $\left\{ a_{2n-1} \right\}_{n=1}^\infty$ both converge to $a$.
\end{theorem}
%
\begin{theorem}[Squeeze Theorem for Sequences]
	Let $\left\{ a_n \right\}_{n=1}^\infty$, $\left\{ b_n \right\}_{n=1}^\infty$, and $\left\{ c_n \right\}_{n=1}^\infty$ be infinite sequences such that $a_n \leq b_n \leq c_n$ for all $n>n_0$. If $\lim_{n\rightarrow\infty}a_n=\lim_{n\rightarrow\infty}c_n=d$, then .
\end{theorem}
%
\subsection{Limits of Sequences}
\begin{theorem}
	Suppose that $\left\{a_n\right\}$ and $\left\{b_n\right\}$ are convergent infinite sequences on $\R$ that converge to $a$ and $b$, respectively, and let $c\in\R$ be a constant, then,
	\begin{enumerate}[label=\alph*)]
		\item $\lim_{n\rightarrow\infty}c=c$.
		\item $\lim_{n\rightarrow\infty}ca_n=ca$.
		\item $\lim_{n\rightarrow\infty}\left( a_n + b_n \right)=a+b$.
		\item $\lim_{n\rightarrow\infty}\left( a_n - b_n \right)=a-b$.
		\item $\lim_{n\rightarrow\infty}\left( a_nb_n \right)=ab$.
		\item $\lim_{n\rightarrow\infty}\left( \frac{a_n}{b_n} \right)=\frac{a}{b}$, ($b\neq 0$).
	\end{enumerate}
\end{theorem}

\section{Series}

%
\subsection{Convergence Tests}

%
\section{Limits of Functions}

\end{document}