%!TEX program = xelatex
\documentclass{article}

\usepackage{geometry}
\geometry{a4paper}

\usepackage{multicol}
\usepackage{changepage}
\usepackage{physics}

\usepackage{amsmath}
\usepackage{amssymb}
\usepackage{amsthm}

% Create nice frames
\usepackage{mdframed}
\mdfsetup{skipabove=\topskip,skipbelow=\topskip}
\mdfdefinestyle{exampledefault}{%
	rightline=true,
	innerleftmargin=10,innerrightmargin=10,
	leftmargin=20,rightmargin=20,
	topline=false,bottomline=false,
	frametitlefont=\bf\large
}

% Tables and figures
\usepackage{booktabs}
\usepackage{tabularx}
\usepackage{float}

\usepackage{enumitem}

% Headers and footers
\usepackage{fancyhdr}
\setlength{\headheight}{15.2pt}
\pagestyle{fancy}
\fancyhf{}

\usepackage[warnings-off={mathtools-colon, mathtools-overbracket}]{unicode-math}
\setmathfont{Latin Modern Math}
\setmathfont{TeX Gyre Pagella Math}[range={bb,bbit}, Scale=MatchUppercase]

%% Headers
\fancyhead[L]{2021 Semester 1}
\fancyhead[C]{MXB102 Exam Notes}
\fancyhead[R]{\thepage}

% Number Sets
\newcommand*{\N}{\mathbb{N}}
\newcommand*{\Z}{\mathbb{Z}}
\newcommand*{\Q}{\mathbb{Q}}
\newcommand*{\I}{\mathbb{I}}
\newcommand*{\R}{\mathbb{R}}
\newcommand*{\C}{\mathbb{C}}

% Table Cell Stretch
\setlength{\tabcolsep}{10pt} % Default value: 6pt
\renewcommand{\arraystretch}{1.5} % Default value: 1

% Set symbols
\let\oldemptyset\emptyset
\let\emptyset\varnothing
\DeclareMathOperator{\setunion}{\cup} % defined for clarity/consistency
\DeclareMathOperator{\setintersection}{\cap}

% Quotation in math mode
\newcommand{\quo}[1]{\text{``}#1\text{"}}

\newcommand{\contradiction}{
    \hspace{-1em}
	{\hbox{
	\setbox0=\hbox{$\mkern-3mu{\times}\mkern-3mu$}
	\setbox1=\hbox to0pt{\hss\copy0\hss}
	\copy0\raisebox{0.5\wd0}{\copy1}\raisebox{-0.5\wd0}{\box1}\box0}}
}

%% Theorem environments
\theoremstyle{plain}
\newtheorem{theorem}{Theorem}[section]
\numberwithin{theorem}{subsection}

\theoremstyle{definition}
\newtheorem{definition}{Definition}[section]
\numberwithin{definition}{subsection}

\theoremstyle{remark}
\newtheorem{note}{Note}[section]
\numberwithin{note}{section}

\newtheorem*{statement}{Statement}

%% Title page
\title{\textbf{Abstract Mathematical Reasoning} \\ {\large Exam Notes} \\ {\normalsize 2021, Semester 1}}
\author{
    Rohan Boas \and Tarang Janawalkar \and Oliver Strong
}
\date{}

\begin{document}

\begin{titlepage}
\maketitle
\thispagestyle{empty}
\end{titlepage}

\newpage

\section{Logic}
\begin{table}[H]
	\centering
	\begin{tabular}{c >{$}c<{$} | c}
			\textbf{Name} & \text{\textbf{Symbol}} & \textbf{Truth Table} \\
			\midrule
		    Negation (NOT) & \neg &
		    \begingroup
		    \renewcommand{\arraystretch}{1}
		    \begin{tabular}{c|c}
		        $P$ & $\neg{P}$ \\
		        \midrule
		        {\sffamily{T}} & {\sffamily{F}} \\
		        {\sffamily{F}} & {\sffamily{T}}
		    \end{tabular}
		    \endgroup
		    \\
		    Conjunction (AND) & \land &
		    \begingroup
		    \renewcommand{\arraystretch}{1}
		    \begin{tabular}{c c|c}
		        $P$ & $Q$ & $P\land Q$ \\
		        \midrule
		        {\sffamily{T}} & {\sffamily{T}} & {\sffamily{T}} \\
		        {\sffamily{T}} & {\sffamily{F}} & {\sffamily{F}} \\
		        {\sffamily{F}} & {\sffamily{T}} & {\sffamily{F}} \\
		        {\sffamily{F}} & {\sffamily{F}} & {\sffamily{F}}
		    \end{tabular}
		    \endgroup
		    \\
		    Disjunction (OR) & \lor &
		    \begingroup
		    \renewcommand{\arraystretch}{1}
		    \begin{tabular}{c c|c}
		        $P$ & $Q$ & $P\lor Q$ \\
		        \midrule
		        {\sffamily{T}} & {\sffamily{T}} & {\sffamily{T}} \\
		        {\sffamily{T}} & {\sffamily{F}} & {\sffamily{T}} \\
		        {\sffamily{F}} & {\sffamily{T}} & {\sffamily{T}} \\
		        {\sffamily{F}} & {\sffamily{F}} & {\sffamily{F}}
		    \end{tabular}
		    \endgroup
		    \\
		    Conditional (IF-THEN) & \implies &
		    \begingroup
		    \renewcommand{\arraystretch}{1}
		    \begin{tabular}{c c|c}
		        $P$ & $Q$ & $P\implies Q$ \\
		        \midrule
		        {\sffamily{T}} & {\sffamily{T}} & {\sffamily{T}} \\
		        {\sffamily{T}} & {\sffamily{F}} & {\sffamily{F}} \\
		        {\sffamily{F}} & {\sffamily{T}} & {\sffamily{T}} \\
		        {\sffamily{F}} & {\sffamily{F}} & {\sffamily{T}}
		    \end{tabular}
		    \endgroup
		    \\
		    Equivalence (IFF) & \iff \text{ or } \equiv &
		    \begingroup
		    \renewcommand{\arraystretch}{1}
		    \begin{tabular}{c c|c}
		        $P$ & $Q$ & $P\iff Q$ \\
		        \midrule
		        {\sffamily{T}} & {\sffamily{T}} & {\sffamily{T}} \\
		        {\sffamily{T}} & {\sffamily{F}} & {\sffamily{F}} \\
		        {\sffamily{F}} & {\sffamily{T}} & {\sffamily{F}} \\
		        {\sffamily{F}} & {\sffamily{F}} & {\sffamily{T}}
		    \end{tabular}
		    \endgroup
	\end{tabular}
	\caption{Boolean operators in order of precedence.}
	\label{tab:Logic Symbols}
\end{table}
\begin{theorem}[de Morgan's Law]
    $\neg{\left(P \land Q\right)} \iff \neg{P} \lor \neg{Q}$.
\end{theorem}
%
\begin{definition}[Tautology]
	A tautology is a statement that is always true.
\end{definition}
%
\begin{definition}[Paradox]
	A paradox is a statement that is always false.
\end{definition}
%
\begin{definition}[Converse]
	The converse of the statement $P\implies Q$	is the statement $Q\implies P$.
\end{definition}
%
\begin{definition}[Contrapositive]
	The contrapositive of $P\implies Q$	is the statement $\neg{Q}\implies \neg{P}$.
\end{definition}
%
\section{Sets}
\begin{table}[H]
	\centering
	\begin{tabular}{c >{$}c<{$} | >{$}c<{$}}
	    \textbf{Name} & \text{\textbf{Symbol}} & \text{\textbf{Definition}} \\
	    \midrule
		Empty set      & \emptyset                & \text{A set with no elements:} \left\{\right\} \\
		Intersection   & \setintersection         & S \setintersection T = \left\{x:x\in S \land x \in T\right\} \\
		Union          & \setunion                & S \setunion T = \left\{x:x\in S \lor x \in T\right\} \\
		Set difference & \backslash \text{ or } - & S - T = \left\{x:x\in S \land x \notin T\right\} \\
		Subset         & \subset                  & \\
		Strict subset  & \subseteq                & \\
		Complement     & \overline{S}  & \text{If } S \subseteq T, \quad \overline{S}=T \backslash S
	\end{tabular}
	% \caption{Set symbols}
	\label{tab:Set Symbols}
\end{table}
%
\begin{definition}[Disjoint]
    If $S \setintersection T = \emptyset$, then $S$ and $T$ are disjoint
    (i.e. two sets are disjoint if their intersection is empty).
\end{definition}
%
\subsection{Construction of the Natural Numbers}
\begin{definition}[$\quo{0}$]
	$\quo{0}$ is defined as the empty set, $\emptyset$ or $\left\{\right\}$, which is the first element of the natural numbers.
\end{definition}
%
\begin{definition}[Successor function]
	The successor function, $S(n)$, is defined as $n \setunion \{n\}$.
\end{definition}
%
\begin{definition}[Addition]
	Addition ($+$) is defined recursively by the following axioms:
	\begin{enumerate}[label={(A\arabic*)}, leftmargin=3.5em, itemsep=0.2em, topsep=0.35em]
		\item $\mathrm{\quo{a} + \quo{0} = \quo{a}}$
		\item $\mathrm{\quo{a} + S(\quo{b}) = S(\quo{a} + \quo{b})}$
	\end{enumerate}
\end{definition}
%
\begin{definition}[Multiplication]
	Multiplication ($\cdot$) is defined recursively by the following axioms:
	\begin{enumerate}[label={(M\arabic*)}, leftmargin=3.5em, itemsep=0.2em, topsep=0.35em]
		\item $\mathrm{\quo{a} \cdot \quo{0} = \quo{0}}$
		\item $\mathrm{\quo{a} \cdot S(\quo{b}) = \quo{a} + (\quo{a} \cdot \quo{b})}$
	\end{enumerate}
\end{definition}
%
\begin{definition}[Exponentiation]
	Exponentiation ($\uparrow$) is defined recursively by the following axioms:
	\begin{enumerate}[label={(E\arabic*)}, leftmargin=3.5em, itemsep=0.2em, topsep=0.35em]
		\item $\mathrm{\quo{a} \uparrow \quo{0} = \quo{1}}$
		\item $\mathrm{\quo{a} \uparrow S(\quo{b}) = \quo{a} \cdot (\quo{a} \uparrow \quo{b})}$
	\end{enumerate}
\end{definition}
%
\subsubsection{Properties of Natural Numbers}
\begin{description}[style=nextline]
	\item[Commutativity of addition]
		$\mathrm{ \quo{a} + \quo{b} = \quo{b} + \quo{a}}$
	\item[Associativity of addition]
		$\mathrm{(\quo{a} + \quo{b}) + \quo{c} = \quo{a} + (\quo{b} + \quo{c})}$
	\item[Commutativity of multiplication]
		$\mathrm{\quo{a} \cdot \quo{b} = \quo{b} \cdot \quo{a}}$
	\item[Associativity of multiplication]
		$\mathrm{(\quo{a} \cdot \quo{b}) \cdot \quo{c} = \quo{a} \cdot (\quo{b} \cdot \quo{c})}$
	\item[Distributivity of multiplication over addition]
		$\mathrm{\quo{a} \cdot  (\quo{b} + \quo{c}) = (\quo{a} \cdot \quo{b}) + (\quo{a} \cdot \quo{c})}$
	\item[Cancellation law]
		If $\mathrm{\quo{a} \cdot \quo{b} = \quo{a} \cdot \quo{c}}$
		and $\mathrm{\quo{a} \ne \quo{0}}$
		then $\mathrm{\quo{b} = \quo{c}}$
\end{description}
%
\subsection{Cartesian Product}
\begin{definition}[Cartesian product]
	The Cartesian product of two sets $S$ and $T$ is the set
	$S\times T$ consisting of all ordered pairs $(x, y)$ where
	$x \in S$ and $y \in T$.
\end{definition}
%
\subsection{Quantifiers}
\begin{table}[H]
    \centering
	\begin{tabular}{c >{$}c<{$} | >{$}c<{$}}
	    \textbf{Name} & \text{\textbf{Symbol}} & \text{\textbf{Usage}} \\
	    \midrule
	     Universal Quantification (For All)        & \forall & \forall x:P(x) \\
	     Existential Quantification (There Exists) & \exists & \exists x:P(x) \\
    \end{tabular}
    % \caption{Quantifiers}
	\label{tab:Quantifiers}
\end{table}
%
\begin{theorem}
$\neg{\left(\forall x:P(x)\right)} \iff \exists x:\neg P(x)$
\end{theorem}
%
\begin{theorem}
$\neg{\left(\exists x:P(x)\right)} \iff \forall x:\neg P(x)$
\end{theorem}
%
\begin{definition}[Greater than ($>$)]
    $x > y \iff \exists a \in \N\backslash\{0\} : x=y+a$
\end{definition}
%
\begin{definition}[Greater than or equal to ($\geqslant$)]
    $x \geqslant y \iff \exists a \in \N : x=y+a$
\end{definition}
%
\section{Relations}
\begin{definition}[Binary relation]
	A binary relation $R$ on $S$ is the subset $S\times S$.
	If $a, \: b \in S$ and  $(a, \: b) \in R$,
	then $aRb$ (``$a$ is related to $b$ under $R$")
\end{definition}
%
\begin{figure}[H]
\begin{mdframed}[style=exampledefault,frametitle={Relation Properties}]
\begin{description}[style=sameline]
	\item[Reflexive:]
		$\forall x \in S : xRx$
	\item[Symmetric:]
		$\forall x,\: y \in S : xRy \implies yRx$
	\item[Antisymmetric:]
		$\forall_\text{distinct} \: x,\: y \in S : xRy \implies \neg (yRx)$
	\item[Transitive:]
		$\forall x,\: y, \: z \in S : (xRy \land yRz) \implies xRz$
\end{description}
\end{mdframed}
\end{figure}
%
\begin{definition}[Equivalence relation]
    A relation is an equivalence relation if it is
    reflexive, symmetric, and transitive.
\end{definition}
\begin{definition}[Functions as rules]
    A function $f$ is a rule which maps
    from some set $S$ (the domain)
    to some set $T$ (the codomain)
    (i.e. $f: S \to T$) where each $x \in S$ maps to a unique $f(x) \in T$.
\end{definition}
\begin{definition}[Image]
    The image of a function $f: S \to T$ is the set $f(S)$,
    which is equal to $\{f(x):x \in S\}$.
\end{definition}
\begin{definition}[Functional]
    A relation is functional if
    $\forall x \in S \land \forall y,\: z \in T
        : (xRy \, \land \, yRz) \implies y=z$.
\end{definition}
\begin{note}
    The functional property corresponds to the vertical line test.
\end{note}
\begin{definition}[Left-total]
    A relation is left-total if
    $\forall x \in S : \exists y \in T : xRy$.
\end{definition}
\begin{definition}[Functions as relations]
    A function $f$ is some relation that is functional and left-total.
\end{definition}
%
\begin{figure}[H]
\begin{mdframed}[style=exampledefault,frametitle={Function properties}]
\begin{description}[style=sameline]
	\item[Injective (one-to-one):]
		$\forall x,\: z \in S : \forall y \in T
		: (xRy \,\land\, zRy) \implies x = z$

		Put in words, every element in the codomain ($T$)
		is being mapped to by \textit{at most} one element from the domain ($S$).
	\item[Surjective (onto):]
		$\forall y \in T : \exists x \in S : xRy$

		Put in words, every element in the codomain ($T$)
		is being mapped to by \textit{at least} one element from the domain ($S$).
	\item[Bijective:]
		$R$ is injective and surjective.
\end{description}
\end{mdframed}
\end{figure}
%
\begin{definition}[Equal functions]
    Two functions are equal if they have the same domain and codomain
    for all $x$, so that $f(x) = g(x)$.
\end{definition}
%
\begin{definition}[Function composition]
    The composition of two functions $f:X \to Y$ and $g:Y \to Z$,
    denoted $g \circ f$, maps $X$ to $Z$, and is defined as
    $(g \circ f)(x) = g(f(x))$.
\end{definition}
%
\begin{theorem}
    For $f:X \to Y$ and $g:Y \to Z$ and  $h:Z \to W$,
    $h \circ (g \circ f) = (h \circ g) \circ f$.
\end{theorem}
%
\begin{definition}[Identity function]
    The identity function of $X$ is $1_X:X \to X$ and is defined as
    $1_X (x)=x$ for all $x \in X$.
\end{definition}
%
\begin{definition}[Inverse function]
    If $f:X \to Y$ and $g:Y \to X$ are functions where
    $g \circ f = 1_X$ and $f \circ g = 1_Y$,
    then $g$ is the inverse of $f$, denoted $f^{-1}$.
\end{definition}
%
\begin{theorem}
    A function $f$ has an inverse if and only if $f$ is bijective.
\end{theorem}
%
\begin{definition}[Equivalence class]
    Let $R$ be an equivalence relation on $S$ and $x \in S$.
    The equivalence class of $x$ is $[x] = \{y \in S : xRy\}$,
    i.e. the set of all elements in $S$ that are related to $x$.

    $x$ is a representative of the equivalence class $[x]$.
    There may be different representations of the same equivalence class.
\end{definition}
%
\section{Number Sets}
\subsection{Rational Numbers}
\begin{definition}[Rational numbers]
    Define an equivalence relation $\sim$ on $\Z \times (\N\backslash\{0\}$
    as $(a,\: b) \sim (c,\: d) \iff ad=bc$.
    The equivalence classes of $\sim$ are rational numbers.
    $[(a,\: b)] = \frac{a}{b}$.
    The set of all rational numbers is
    $\Q = \left\{ \frac{a}{b} : a\in \Z, b\in \N_{>0} \right\}$.
\end{definition}
%
\begin{definition}[Addition on rational numbers]
    $(a,\: b) + (c,\: d) = (ad+bc,\: bd)$
\end{definition}
%
\begin{definition}[Multiplication on rational numbers]
    $(a,\: b) \cdot (c,\: d) = (ac,\: bd)$
\end{definition}
%
\section{Algebraic Structures}
\begin{definition}[Commutativite ring with identity]
	A commutative ring with identity is a set $S$ with two binary operations $+$ and $\cdot$ such that the following ring axioms are satisfied.
\end{definition}
\begin{figure}[H]
	\begin{mdframed}[style=exampledefault,frametitle={Ring Axioms}]
		\begin{enumerate}[leftmargin=3.5em, itemsep=0.2em, topsep=0.35em]
			\item [(C1)] $+$ is closed: $\forall a,\:b \in S: a+b \in S$.
			\item [(A1)] $+$ is associative: $\forall a,\:b,\:c \in S: (a+b)+c = a+(b+c)$.
			\item [(A2)] $+$ is commutative: $\forall a,\:b \in S: a+b=b+a$.
			\item [(A3)] additive identity: $\forall a\in S:\exists z \in S:a+z=a$.
			\item [(A4)] additive inverse: $\forall a \in S, \exists b\in S: a+b = z$ (denoted ``0").
			\item [(C2)] $\cdot$ is closed: $\forall a,\:b \in S: a \cdot b \in S$.
			\item [(M1)] $\cdot$ is associative: $\forall a,\:b,\:c \in S:a \cdot (b \cdot c) = (a \cdot c) \cdot c$.
			\item [(M2)] $\cdot$ is commutative: $\forall a,\:b \in S:a \cdot b = b \cdot a$.
			\item [(M3)] multiplicative identity: $\forall a \in S:\exists e \in S:a \cdot e = a$ (denoted ``1").
			\item [(D)] distributivity: $\forall a,\:b,\:c \in S: a \cdot (b + c) = (a \cdot b) + (a \cdot c)$.
		\end{enumerate}
	\end{mdframed}
	\begin{note}
		$\Z$, $\Q$, $\R$, and $\C$ are rings.
	\end{note}
\end{figure}
% \marginpar{\raggedright\footnotesize The definition of rings and fields has been excluded from this document for brevity}
%
\begin{definition}[Integers]
    The set of integers ($\Z$) is the smallest ring containing the natural numbers.
\end{definition}
\begin{note}
    The integers are the closure of the natural numbers with respect to the ring axioms.
\end{note}
%
\begin{definition}[Real numbers]
    Let $D = \Z \times \{\text{sequence of digits $0\ldots 9$}\}$.
    Define an equivalence relation on $D$ such that
    \begin{equation*}
		x_0 . x_1 x_2 \dots x_{k-1} x_k \overline{000} = x_0 . x_1 x_2 \dots x_{k-1} \left(x_k - 1\right)\overline{999}
	\end{equation*}
    The real numbers ($\R$) is the set of the equivalence classes of D using the equivalence relation above.
\end{definition}
%
\begin{definition}[Irrational numbers]
    The irrational numbers ($\I$ or $\overline{\Q}$) are defined as $\R\backslash\Q$.
\end{definition}
%
\begin{definition}[Field]
	Let $S$ be a set that satisfies the ring axioms. $S$ is also a field if it satisfies the following axiom.
	\begin{enumerate}
		\item [(M4)] multiplicative inverse: $\forall a \in S \ \{0\}:\exists b \in S: a\cdot b=e$.
	\end{enumerate}
	\begin{note}
		$\Q$, $\R$, and $\C$ are rings.
	\end{note}
\end{definition}
%
\section{Cardinality}
\begin{definition}[Cardinality]
    The cardinality of $X$ ($\#X$ or $\abs{X}$), is the number of elements in $X$.
    If there is a bijection between two sets $X$ and $Y$,
    then $\#X=\#Y$ or equivalently, $\abs{X} = \abs{Y}$.
\end{definition}
%
\begin{definition}[$\N_{<n}$]
    The subset of $\N$ containing all naturals less than $n$.
\end{definition}
%
\begin{definition}[Finite and infinite sets]
    If a set $X$ is such that $\#X=\#\N_{<n}$,
    then the number of elements in $X$ is $n$ and $X$ is a finite set.
    If there is no $n$ such that there is a bijection between $X$ and $\N_{<n}$,
    then $X$ is an infinite set.
\end{definition}
%
\begin{definition}[Countable and uncountable infinities]
    If there exists a bijection between an infinite set $X$ and $\N$,
    then $X$ is countably infinite, else uncountably infinite.
\end{definition}
%
\begin{theorem}$\#\N = \#\Z$\end{theorem}
\begin{theorem}$\#(\N\times\N) = \#\N$\end{theorem}
\begin{theorem}$\#\Q = \#\N$\end{theorem}
\begin{theorem}$\#\N \ne \#\R$\end{theorem}
%
\begin{definition}[Power set]
    The power set of $X$, $\mathscr{P}(X)$, is the set of all subsets of $X$,
    including $\emptyset$ and $X$.
\end{definition}
\begin{theorem}
    The cardinatility of $\mathscr{P}(X)$ is $2^{\abs{X}}$.
\end{theorem}
%
\section{Proofs}
\begin{figure}[H]
	\begin{mdframed}[style=exampledefault,frametitle={Proof Structure}]
		\begin{enumerate}[leftmargin=3.5em, itemsep=0.2em, topsep=0.35em]
			\item State the proof method that will be used.
			\item If the proof method has conditions show that these conditions have been satisfied before using the proof method.
			\item Clearly state any assumptions and definitions required by the proof.
			\item Apply the proof method.
			\item Write a sentence summarising the proof method.
			\item Conclude the proof with QED, $\square$, or $\blacksquare$.
		\end{enumerate}
	\end{mdframed}
\end{figure}
%
\subsection{Direct Proof}
\begin{enumerate}
    \item Use a definition (even/odd numbers).
    \item Show LHS = RHS.
    \item For an equivalence ($\iff$) statement, show that both forward ($\implies$) and backward ($\impliedby$) directions are true.
\end{enumerate}
%
\subsection{Contradiction}
\begin{enumerate}
    \item Assume that the statement is false.
    \item Make a sequence of logical statements that follow from the assumption.
    \item Arrive at a contradiction.
    \item Since we arrive at a contradiction, the original assumption must have been wrong.
    \item Hence the original statement must be true.
    \item The $\contradiction$ and $\implies\!\!\!\!\impliedby$ symbols
            are often used to show a contradiction.
\end{enumerate}
%
\subsection{Contrapositive}
\begin{enumerate}
    \item Write the original statement as a conditional statement.
    \item Rewrite the statement in the contrapositive form.
    \item Try to prove the contrapositive statement, usually a direct proof.
    \item By proving the contrapositive, you have also proved the original statement.
\end{enumerate}
%
\subsection{Contradiction vs. Contrapositive}
For the conditional statement: ``If $P$, then $Q$".

\textbf{By contradiction:} Assume $P$ and $\neg{Q}$, and find some contradiction.

\textbf{By contrapositive:} Assume $\neg{Q}$, and show $\neg{P}$.
%
\subsection{Induction}
\begin{enumerate}
    \item Find the logical statement $P(n)$ so that the theorem can be written in the form: ``Show $P(n)$ is true for all $n \in \N_{\geqslant n_0}$".
    \item Prove the base case is true: $P(n)$ is true for $n=n_0$.
    \item Write the inductive hypothesis: ``Assume $P(n)$ is true for $n=k$".
    \item Write the inductive step: ``$P(n)$ is true for $n=k+1$".
    \item Prove the inductive step is true, usually a direct proof. Remember to utilise the inductive hypothesis.
    \item Conclusion: Since $P(n_0)$ is true, and $P(k+1)$ is true whenever $P(k)$ is true, then the principle of mathematical induction implies $P(n)$ is true for all $n\in \N_{\geqslant n_0}$.
\end{enumerate}
%
\subsection{Strong Induction}
\begin{enumerate}
    \item Find the logical statement $P(n)$ so that the theorem can be written in the form: ``Show $P(n)$ is true for all $n \in \N_{\geqslant n_0}$".
    \item Prove the base case is true: $P(n)$ is true for $n=n_0$.
    \item Write the inductive hypothesis: ``Assume $P(n)$ is true for $n\leqslant k$".
    \item Write the inductive step: ``$P(n)$ is true for $n=k+1$".
    \item Prove the inductive step is true, usually a direct proof. Remember to utilise the inductive hypothesis.
    \item Conclusion: Since $P(n_0)$ is true, and $P(k+1)$ is true whenever $P(n_0),\: \ldots,\: P(k)$ is true, then the principle of strong mathematical induction implies $P(n)$ is true for all $n\in \N_{\geqslant n_0}$.
\end{enumerate}
%
\subsection{Useful Techniques}
\begin{enumerate}
    \item Use the definition of the numbers, i.e. even and odd numbers.
    \item For rational numbers assume that $(a,b)$ is fully simplified, so that $a$ and $b$ are co-prime.
    \item When using strong induction, remember that $P(k), \ldots, P(n_0)$ are assumed to be true.
    \item $a^b=e^{\ln{a^b}}=e^{b\ln{a}}$.
\end{enumerate}
%
\section{Sequences}
An infinite sequence on $X$ is a function $a$ with domain $\N_{> 0}$ and codomain $X$. That is, $a:\N_{> 0}\to X$.

For infinite sequences we write $a_n$ and the sequence can be represented as $\left\{a_n\right\}_{n=1}^\infty$, where $n$ is the index of the sequence.
%
\subsection{Convergence}
\begin{theorem}
	A sequence $\left\{a_n\right\}_{n=1}^\infty$ converges to $a\in \R$ if
	\begin{equation*}
		\forall \varepsilon > 0 : \exists n_0 \in \N_{> 0}:\forall n \geqslant n_0 : \abs{a_n - a} < \varepsilon
	\end{equation*}
	where $\lim_{n \to \infty}a_n = a$.
\end{theorem}
%
\begin{theorem}
	A sequence $\left\{a_n\right\}_{n=1}^\infty$ diverges to $+\infty$ if
	\begin{equation*}
		\forall N \in \R : \exists n_0 \in \N_{> 0}:\forall n \geqslant n_0 : a_n > N
	\end{equation*}
	where $\lim_{n \to \infty}a_n = \infty$.
\end{theorem}
%
\begin{theorem}
	A sequence $\left\{a_n\right\}_{n=1}^\infty$ diverges to $-\infty$ if
	\begin{equation*}
		\forall M \in \R : \exists n_0 \in \N_{> 0}:\forall n \geqslant n_0 : a_n < M
	\end{equation*}
	where $\lim_{n \to \infty}a_n = -\infty$.
\end{theorem}
%
\begin{theorem}[Triangle inequality]
	$\abs{a+b}\leqslant\abs{a}+\abs{b}$.
\end{theorem}
%
\begin{theorem}
	A sequence $\left\{ a_n \right\}_{n=1}^\infty$ converges to $a$ if and only if the sequences $\left\{ a_{2n} \right\}_{n=1}^\infty$ and $\left\{ a_{2n-1} \right\}_{n=1}^\infty$ both converge to $a$.
\end{theorem}
%
\begin{theorem}[Squeeze theorem for sequences]
	Let $\left\{ a_n \right\}_{n=1}^\infty$, $\left\{ b_n \right\}_{n=1}^\infty$, and $\left\{ c_n \right\}_{n=1}^\infty$ be infinite sequences such that $a_n \leqslant b_n \leqslant c_n$ for all $n>n_0$. If $\lim_{n\to\infty}a_n=\lim_{n\to\infty}c_n=d$, then .
\end{theorem}
%
\begin{figure}[H]
	\begin{mdframed}[style=exampledefault,frametitle={Limits of Sequences}]
		\begin{theorem}
			Suppose that $\left\{a_n\right\}$ and $\left\{b_n\right\}$ are convergent infinite sequences on $\R$ that converge to $a$ and $b$, respectively, and let $c\in\R$ be a constant, then,
			\begin{enumerate}[label=\normalfont\alph*)]
				\item $\lim_{n\to\infty}c=c$.
				\item $\lim_{n\to\infty}ca_n=ca$.
				\item $\lim_{n\to\infty}\left( a_n + b_n \right)=a+b$.
				\item $\lim_{n\to\infty}\left( a_n - b_n \right)=a-b$.
				\item $\lim_{n\to\infty}\left( a_nb_n \right)=ab$.
				\item $\lim_{n\to\infty}\left( \frac{a_n}{b_n} \right)=\frac{a}{b}$, ($b\neq 0$).
			\end{enumerate}
		\end{theorem}
	\end{mdframed}
\end{figure}
%
\subsection{Cauchy Sequences}
\subsection{Monotonicity}
\section{Series}
%
\subsection{Convergence Tests}

%
\section{Limits of Functions}
\begin{definition}[Finite limit ($\varepsilon$-$\delta$ ver.)]
    Let a function $f(x)$ be defined for all $x$ in an open interval $I$
    (which contains $x_0 \in \R$), except for $x_0$ which may or may not be defined.
    $f(x)$ has the limit $L \in \R$ at $x_0$ if
    $\forall\varepsilon>0: \exists\delta>0: \forall x \in I: 0<\abs{x-x_0}<\delta
    \implies \abs{f(x)-L}<\varepsilon$
    $\lim_{x\to x_0} f(x) = L$
\end{definition}
%
\begin{figure}[H]
\begin{mdframed}[style=exampledefault,frametitle={Limit laws}]
    \begin{theorem} The limit of a sum equals the sum of the limits, i.e.
        \begin{equation*}
              \lim_{x\to x_0}\left(f(x)+g(x)\right)
            = \left(\lim_{x\to x_0}f(x)\right) + \left(\lim_{x\to x_0}g(x)\right)
        \end{equation*}
    \end{theorem}
    \begin{theorem} The limit of a product equals the product of the limits, i.e.
        \begin{equation*}
          \lim_{x\to x_0}\left(f(x)\cdot g(x)\right)
        = \left(\lim_{x\to x_0}f(x)\right) \cdot \left(\lim_{x\to x_0}g(x)\right)
        \end{equation*}
    \end{theorem}

        These are trivially extended to
        division (if the divisor is non-zero),
        subtraction, exponentiation, and moving constants in or outside of a limit, i.e.
        \begin{equation*}
          \lim_{x\to x_0}\left(f(x) - g(x)\right)
        = \left(\lim_{x\to x_0}f(x)\right) - \left(\lim_{x\to x_0}g(x)\right)
        \end{equation*}
        \begin{equation*}
        \lim_{x\to x_0}\frac{f(x)}{g(x)}
        = \frac{\lim_{x\to x_0}f(x)}{\lim_{x\to x_0}g(x)}
        \end{equation*}
        \begin{equation*}
        \lim_{x\to x_0}\left(f(x)\right)^n
        = \left(\lim_{x\to x_0}f(x)\right)^n
        \end{equation*}
        \begin{equation*}
        \lim_{x\to x_0} c \cdot f(x)
        = c \cdot \lim_{x\to x_0}f(x)
        \end{equation*}
\end{mdframed}
\end{figure}
%
\begin{definition}[Limit of $\pm\infty$ ($\varepsilon$-$\delta$ ver.)]
    Let a function $f(x)$ be defined for all $x$ in an open interval $I$
    (which contains $x_0 \in \R$), except for $x_0$ which may or may not be defined.

    $f(x)$ has the limit $+\infty$ at $x_0$
    $\left(\displaystyle\lim_{x\to x_0} f(x) = +\infty\right)$ if
    $ \forall M \in \R: \exists\delta>0: \forall x \in I: 0<\abs{x-x_0}<\delta
    \implies f(x) > M $

    $f(x)$ has the limit $-\infty$ at $x_0$
    $\left(\displaystyle\lim_{x\to x_0} f(x) = -\infty\right)$ if
        $\forall N \in \R: \exists\delta>0: \forall x \in I: 0<\abs{x-x_0}<\delta
            \implies f(x) < N$
\end{definition}
%
\begin{definition}[Limits towards infinity]
    Let a function $f(x)$ be defined for all $x$ in an infinite open interval $I$
    extending in the positive $x$ direction.
    $\displaystyle \lim_{x\to\infty} f(x) = L$ if
    $\forall \varepsilon>0: \exists M \in I: x>M
        \implies \abs{f(x)-L}<\varepsilon$

    Let a function $g(x)$ be defined for all $x$ in an infinite open interval $I$
    extending in the negative $x$ direction.
    $\displaystyle \lim_{x\to-\infty} g(x) = L$ if
    $\forall \varepsilon>0: \exists N \in I: x<N
        \implies \abs{g(x)-L}<\varepsilon$
\end{definition}
%
\begin{theorem}
    $\displaystyle\lim_{x\to x_0} f(x)$ exists iff
    $\displaystyle\lim_{x\to {x_0}^+} f(x)$ and $\displaystyle\lim_{x\to {x_0}^-} f(x)$
    exist and are equal
\end{theorem}
%
\begin{theorem}[Squeeze theorem for functions]
    Let an interval $I$ contain $x_0$.
    If $f,\:g,\:h : I \to \R$ are functions such that
    $\forall x \in I\backslash\{x_0\},\; f(x)\leqslant g(x)\leqslant h(x)$
    and
    $\displaystyle\lim_{x\to x_0} f(x) = \lim_{x\to x_0} h(x) = L$,
    then
    $\displaystyle\lim_{x\to x_0} g(x) = L$.
\end{theorem}
%
\begin{note}
   The squeeze theorem holds for left- and right-handed and limits to $\pm\infty$.
\end{note}
%
\begin{definition}[Function continuity ($\varepsilon$-$\delta$ ver.)]
    A function $f(x):I\to\R$ is continuous at $c \in I$ if
    $\forall\varepsilon>0: \exists\delta>0: \forall x\in I:
        0<\abs{x-c}<\delta \implies \abs{f(x)-f(c)}<\varepsilon$.

    $f(x)$ is continuous on $I$ if $f(x)$ is continuous for all $x \in I$.
\end{definition}
%
\begin{definition}[Function continuity (sequence ver.)]
    A function $f(x):I\to\R$ is continuous at $c \in I$ if
    for every sequence $\{a_n\}^{\infty}_{n=1}$ in $I$ that converges to $c$
    $\{f(a_n)\}^{\infty}_{n=1}$ converges to $f(c)$.
\end{definition}
%
\begin{note}
    ``$f(x)$ is continuous at $c$'' is equivalent to
    ``$\displaystyle \lim_{x\to c} f(x) = f(c)$''.
\end{note}
%
\begin{note}
    Polynomials are continuous everywhere.
\end{note}
%
\begin{note}
    $\sin$ and $\cos$ are continuous everywhere.
\end{note}
%
\begin{theorem}
    For a function $f$ continuous at $L$
    and another function $g$ where $\displaystyle \lim_{x\to c}g(x) = L$,
    $\displaystyle \lim_{x\to c} (f \circ g)(x) = f(L)$.
\end{theorem}
\begin{note}
    The above theorem still applies when $c$ is $c^\pm$ or $\pm\infty$.
\end{note}
%
\begin{theorem}
    For two functions $f$ and $g$ that are continuous at $c$,
    \begin{enumerate}[label=\normalfont\alph*)]
        \item $f(x) + g(x)$ is continuous at $c$;
        \item $f(x) - g(x)$ is continuous at $c$;
        \item $f(x) \cdot g(x)$ is continuous at $c$; and
        \item $\dfrac{f(x)}{g(x)}$ is continuous at $c$ when $g(c)\ne0$.
    \end{enumerate}
\end{theorem}
%
\begin{theorem}
    For a function $g(x)$ which is continuous at $c$ and
    a function $h(x)$ which is continuous at $g(c)$,
    $h \circ g$ is continuous at $c$.
\end{theorem}
%
\begin{theorem}[L'Ho\^pital rule]
    For two differentiable (and therefore continuous) functions $f(x)$ and $g(x)$
    except for possibly at $x_0$,
    if $\displaystyle \lim_{x\to x_0}f(x)=\lim_{x\to x_0}g(x)=0$,
    or $\displaystyle \lim_{x\to x_0}f(x)=\pm\infty$
        and $\displaystyle \lim_{x\to x_0}g(x)=\pm\infty$,
    then
    $\lim_{x\to x_0}\frac{f(x)}{g(x)} = \lim_{x\to x_0}\frac{f'(x)}{g'(x)}$
    (as long as the limit exists or is $\pm\infty$).
\end{theorem}
%
\begin{note}
    The L'Ho\^pital rule also holds for left- and right-handed and limits to $\pm\infty$.
\end{note}

\end{document}
